\begin{abstractTH}
    
    โครงงานนี้มีชื่อว่า "CapYaDoo" ซึ่งเป็นแอปพลิเคชันบนโทรศัพท์มือถือระบบปฏิบัติการ Android ที่ออกแบบมาเพื่อช่วยเหลือผู้สูงอายุในการจัดการยา โดยระบบสามารถ สแกนฉลากยา หรือ บาร์โค้ด เพื่อค้นหาข้อมูลยาได้อย่างสะดวกและรวดเร็ว ผู้ใช้งานสามารถ บันทึกยาที่กำลังใช้อยู่, จดบันทึกอาการหลังการรับประทานยา, และ ตั้งการแจ้งเตือนเพื่อเตือนเวลาทานยา ได้ภายในแอปเดียว
    
    ระบบพัฒนาด้วย Flutter (Dart) สำหรับฝั่ง Mobile Application และ Spring Boot ร่วมกับ PostgreSQL สำหรับฝั่งเซิร์ฟเวอร์ มีการประมวลผลข้อมูลยาจากฐานข้อมูล TMT (Thai Medicine Terminology) โดยใช้ PySpark สำหรับจัดการและเตรียมข้อมูลก่อนนำเข้าฐานข้อมูล และทำการ Deploy ระบบโดยใช้ Kind เพื่อจัดการสภาพแวดล้อม Container
    
\end{abstractTH}

