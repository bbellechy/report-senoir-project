\begin{abstractTH}
    
    โครงงานนี้จัดทำแอปพลิเคชันสำหรับสแกนฉลากยาและการแจ้งเตือนการทานยา โดยมีชื่อแอปพลิเคชันว่า CapYaDoo ซึ่งเป็นแอปพลิเคชันบนโทรศัพท์มือถือระบบปฏิบัติการ Android ที่ออกแบบมาเพื่อช่วยเหลือผู้สูงอายุในการจัดการยา โดยระบบสามารถ สแกนฉลากยา หรือ บาร์โค้ด เพื่อค้นหาข้อมูลยาได้อย่างสะดวกและรวดเร็ว ผู้ใช้งานสามารถ บันทึกยาที่กำลังใช้อยู่, จดบันทึกอาการหลังการรับประทานยา และ ตั้งการแจ้งเตือนเพื่อเตือนเวลาทานยา ได้ภายในแอปพลิเคชันเดียว
    
    ระบบพัฒนาด้วย Flutter สำหรับฝั่ง Mobile Application และ Spring Boot ร่วมกับ PostgreSQL สำหรับฝั่งเซิร์ฟเวอร์ มีการประมวลผลข้อมูลยาจากสำนักพัฒนามาตรฐานระบบข้อมูลสุขภาพไทย ผ่านระบบ Thai Medicine Terminology (TMT) และข้อมูลจากสำนักงานคณะกรรมการอาหารและยา (อย.) ซึ่งเป็นแหล่งข้อมูลเปิดที่ให้บริการรายละเอียดเกี่ยวกับยา โดยใช้ PySpark สำหรับจัดการและเตรียมข้อมูลก่อนนำเข้าฐานข้อมูล และทำการ Deploy ระบบโดยใช้ Kind ซึ่งเป็นเครื่องมือจำลอง Kubernetes บนเครื่อง เพื่อจัดการสภาพแวดล้อม Container
    
\end{abstractTH}

