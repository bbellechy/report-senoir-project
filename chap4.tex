\chapter{ระเบียบวิธีดำเนินโครงงาน}
\label{Ch:Methodology}

\section{การวิเคราะห์และออกแบบระบบ}
\label{sec:system-analysis-design}

\subsection{Persona}
\label{subsec:persona}
หมายถึง ตัวตนสมมติที่เป็นตัวแทนกลุ่มลูกค้าเป้าหมายของธุรกิจ โดยอิงจากข้อมูลการวิจัยและวิเคราะห์พฤติกรรม ความต้องการ และแรงจูงใจของกลุ่มลูกค้าจริง เพื่อให้ธุรกิจเข้าใจลูกค้าได้อย่างลึกซึ้ง \cite{persona2023}
\subsubsection{ความสำคัญของ Persona}
\begin{enumerate}[leftmargin=4em]
    \item ช่วยให้ธุรกิจเข้าใจลูกค้าได้อย่างลึกซึ้ง
    \item ปรับกลยุทธ์ทางการตลาดให้เหมาะสมกับพฤติกรรมของกลุ่มเป้าหมาย
    \item เพื่อให้สามารถสื่อสารได้อย่างตรงใจลูกค้า
\end{enumerate}

โดยมีตัวอย่างของ Persona ดังนี้
\begin{figure}[!htb]
    \centering
    \includegraphics[width=0.85\textwidth]{figures/chap4/persona/persona_1.png}
    \captionsetup{name=รูป}
    \caption{ตัวตนสมมติที่เป็นตัวแทนของกลุ่มผู้ใช้เป้าหมายคนที่ 1}
    \label{fig:persona_1}
\end{figure}

\begin{figure}[!htb]
    \centering
    \includegraphics[width=0.85\textwidth]{figures/chap4/persona/persona_2.png}
    \captionsetup{name=รูป}
    \caption{ตัวตนสมมติที่เป็นตัวแทนของกลุ่มผู้ใช้เป้าหมายคนที่ 2}
    \label{fig:persona_2}
\end{figure}

\begin{figure}[!htb]
    \centering
    \includegraphics[width=0.85\textwidth]{figures/chap4/persona/persona_3.png}
    \captionsetup{name=รูป}
    \caption{ตัวตนสมมติที่เป็นตัวแทนของกลุ่มผู้ใช้เป้าหมายคนที่ 3}
    \label{fig:persona_3}
\end{figure}

\clearpage


\subsection{User Journey Map}
\label{subsec:user-journey-map}
\vspace*{0pt}
หรือ แผนภาพเส้นทางผู้ใช้ เป็นเครื่องมือที่ใช้อธิบายขั้นตอนและประสบการณ์ของผู้ใช้เมื่อมีปฏิสัมพันธ์กับแอปพลิเคชันของเรา ว่าผู้ใช้โต้ตอบอย่างไร พบปัญหาตรงไหน และจะปรับปรุงจุดใดได้บ้าง เพื่อให้ผู้ใช้ได้รับประสบการณ์ที่ดีขึ้น ซึ่งจะแสดงภาพรวมตั้งแต่จุดเริ่มต้นที่ผู้ใช้เริ่มสนใจไปจนถึงการบรรลุเป้าหมายโดยแบ่งเป็นช่วงเวลาและเหตุการณ์สำคัญต่าง ๆ \cite{journey}
โดยมีตัวอย่างของ User Journey Map ดังนี้
%  โดยมีองค์ประกอบ ดังนี้

% \begin{enumerate}[leftmargin=2em]
%     \item \textbf{The User:} ผู้ใช้เป้าหมาย แรงจูงใจ และปัญหาที่พบ
%     \item \textbf{The Scenario \& Objective:} สถานการณ์และวัตถุประสงค์เพื่ออธิบายว่าผู้ใช้กำลังพยายามทำอะไร
%     \item \textbf{Journey Phases:} ช่วงเส้นทางของผู้ใช้ โดยจะแบ่งออกเป็น 5 ช่วงหลัก ได้แก่
%     \begin{enumerate}
%         \item \textbf{Awareness:} เริ่มรู้จักผลิตภัณฑ์
%         \item \textbf{Consideration:} พิจารณา
%         \item \textbf{Decision:} ตัดสินใจ
%         \item \textbf{Purchase:} ซื้อหรือใช้งาน
%         \item \textbf{Retention:} กลับมาใช้อีก
%     \end{enumerate}
%     \item \textbf{Actions, Attitudes, and Emotions:} พฤติกรรมและอารมณ์ของผู้ใช้ในแต่ละช่วง
%     \item \textbf{Opportunities:} โอกาสในการปรับปรุง วิเคราะห์ข้อมูลที่ได้เพื่อหาทางปรับปรุง
% \end{enumerate}

\begin{figure}[!htb]
    \centering
    \includegraphics[width=0.95\textwidth]{figures/chap4/user_journey/journey_1.png}
    \captionsetup{name=รูป}
    \caption{ประสบการณ์ของผู้ใช้เมื่อมีปฏิสัมพันธ์กับแอปพลิเคชันของเราคนที่ 1}
    \label{fig:journey_1}
\end{figure}

\begin{figure}[!htb]
    \centering
    \includegraphics[width=0.95\textwidth]{figures/chap4/user_journey/journey_2.png}
    \captionsetup{name=รูป}
    \caption{ประสบการณ์ของผู้ใช้เมื่อมีปฏิสัมพันธ์กับแอปพลิเคชันของเราคนที่ 2}
    \label{fig:journey_2}
\end{figure}

\begin{figure}[!htb]
    \centering
    \includegraphics[width=0.95\textwidth]{figures/chap4/user_journey/journey_3.png}
    \captionsetup{name=รูป}
    \caption{ประสบการณ์ของผู้ใช้เมื่อมีปฏิสัมพันธ์กับแอปพลิเคชันของเราคนที่ 3}
    \label{fig:journey_3}
\end{figure}

\subsection{ปัญหาและแนวทางการแก้ไข}
\label{subsec:problems-solutions}

\begin{enumerate}[leftmargin=4em]
    \item \textbf{ปัญหาด้านการจัดการยาหลายชนิด:} ผู้สูงอายุมักต้องทานยาหลายชนิดในแต่ละวัน ทำให้สับสนและลืมรับประทาน\\
    \textbf{แนวทางแก้ไข:}
    \begin{itemize}[leftmargin=2em]
        \item พัฒนาแอปที่ช่วยจัดการยาตามเวลาอย่างเป็นระบบ
        \item มีฟังก์ชันบันทึกและตรวจสอบการทานยา
    \end{itemize}

    \item \textbf{ปัญหาด้านการขาดข้อมูลยาที่ครบถ้วนและน่าเชื่อถือ: } ผู้ใช้ต้องการข้อมูลเกี่ยวกับยา เช่น ผลข้างเคียง\\
    \textbf{แนวทางแก้ไข:}
    \begin{itemize}[leftmargin=2em]
        \item นำเข้าข้อมูลยาจากแหล่งอ้างอิงที่เชื่อถือได้
        \item แสดงข้อมูลอย่างชัดเจน และสามารถเข้าถึงง่าย เช่น ผ่านข้อความหรือเสียงอ่าน
    \end{itemize}

    \item \textbf{ปัญหาด้านการแจ้งเตือนและการติดตามการทานยา: } ผู้สูงอายุต้องการระบบแจ้งเตือนที่ช่วยให้ไม่ลืมทานยา และสามารถบันทึกข้อมูลการทานยาได้\\
    \textbf{แนวทางแก้ไข:}
    \begin{itemize}[leftmargin=2em]
        \item ระบบแจ้งเตือนยาที่ใช้งานง่าย พร้อมสัญลักษณ์หรือเสียง
        \item ระบบที่สามารถบันทึกการทานยาได้
    \end{itemize}
\end{enumerate}

\subsection{การวิเคราะห์ความต้องการ (Requirement Analysis)}
\label{subsec:requirement-analysis}

\subsubsection{ความต้องการของผู้ใช้ (User Requirements)}

\begin{enumerate}[leftmargin=4em]
    \item ระบบจัดการข้อมูลยา (เพิ่ม/แก้ไข/ลบ)
    \item ระบบแจ้งเตือนการทานยาที่ตั้งเวลาได้
    \item ระบบการสแกนข้อมูลยาจากฉลาก (Barcode \& OCR)
    \item ระบบบันทึกอาการหลังการใช้ยา
    \item ระบบอ่านข้อมูลยาให้ผู้ใช้ฟัง (Text-to-Speech)
\end{enumerate}

\subsubsection{ความต้องการของระบบ (System Requirements)}

\begin{enumerate}[leftmargin=4em]
    \item สามารถทำงานได้บน Android
    \item มีความเร็วและความแม่นยำในการสแกนข้อมูล
    \item มีการจัดเก็บข้อมูลผู้ใช้และข้อมูลยาอย่างปลอดภัย
    \item สามารถแสดงข้อมูลได้อย่างครบถ้วน ไม่ตกหล่น
\end{enumerate}

\subsection{การออกแบบสถาปัตยกรรม (System Architecture Design)}
\label{subsec:system-architecture}

\begin{enumerate}[leftmargin=4em]
\item \textbf{Source:}
ข้อมูลต้นทางมาจาก สำนักพัฒนามาตรฐานระบบข้อมูลสุขภาพไทย ซึ่งเผยแพร่ข้อมูลในรูปแบบไฟล์ Excel และจากเว็บไซต์ของทางสำนักงานคณะกรรมการอาหารและยา โดยข้อมูลดังกล่าวจะถูกนำมาใช้เป็นแหล่งข้อมูลหลักของระบบ

\item \textbf{Data Processing:}
ข้อมูลจากไฟล์ Excel จะถูกนำเข้าสู่กระบวนการประมวลผลด้วย PySpark ซึ่งเป็นเครื่องมือที่เหมาะสมสำหรับการจัดการและประมวลผลข้อมูลขนาดใหญ่ (Big Data Processing) โดยจะทำการจัดรูปแบบข้อมูลให้อยู่ในโครงสร้างที่เหมาะสม จากนั้นนำชื่อยาที่ได้มาทำการอ้างอิงเพื่อค้นหาข้อมูลเพิ่มเติมจากเว็บไซต์ของสำนักงานคณะกรรมการอาหารและยา เพื่อใช้เป็นข้อมูลประกอบสำหรับแอปพลิเคชัน

\item \textbf{Database:}
หลังจากผ่านการประมวลผลแล้ว ข้อมูลจะถูกโหลดเข้าสู่ฐานข้อมูล PostgreSQL เพื่อจัดเก็บอย่างเป็นระบบและสามารถเข้าถึงได้อย่างมีประสิทธิภาพ

\item \textbf{Backend Service:}
ใช้ Spring Boot Framework ในการพัฒนาเป็นบริการฝั่งเซิร์ฟเวอร์ (Backend Service) ทำหน้าที่เป็นตัวกลางระหว่างฐานข้อมูลและฝั่งผู้ใช้งาน โดยจัดการคำขอ (Request) จากฝั่ง Client รวมถึงประมวลผลและส่งข้อมูลกลับไปยังผู้ใช้งานในรูปแบบ API

\item \textbf{Client Application:}
ส่วนของผู้ใช้งาน (Frontend) ถูกพัฒนาด้วย Flutter ซึ่งเป็นเครื่องมือสำหรับพัฒนาแอปพลิเคชันแบบ Cross-platform โดยในโครงงานนี้จะพัฒนาแอปพลิเคชันแบบ Android เท่านั้น

\item \textbf{System Deployment:}
สำหรับการทดสอบและจำลองสภาพแวดล้อมการทำงานของระบบบน Kubernetes โดยมีการใช้เครื่องมือ kind (Kubernetes IN Docker) เพื่อสร้าง Kubernetes cluster จำลองบนเครื่องพัฒนา (Local Environment) โดย cluster ที่สร้างขึ้นจะประกอบด้วย container หลัก ได้แก่ Spring Boot Application Container, PySpark Container และ PostgreSQL Database Container

% ซึ่งเชื่อมต่อกันผ่าน Kubernetes service ภายใน cluster

% การใช้ kind ช่วยให้สามารถจำลองขั้นตอนการ deploy ระบบจริง เช่น การสร้าง Pod, Service, และการ expose port ให้สามารถเข้าถึงได้จากภายนอก เหมือนการใช้งานบนสภาพแวดล้อม Cloud จริง ทั้งยังช่วยลดค่าใช้จ่ายและเวลาในการเตรียมระบบทดสอบอีกด้วย

\begin{figure}[H]
    \centering
    \includegraphics[width=0.9\textwidth]{figures/chap4/architecture/architecture.png}
    \captionsetup{name=รูป}
    \caption{System Architecture Design}
    \label{fig:system_architecture}
\end{figure}
\end{enumerate}



% \subsection{การออกแบบฐานข้อมูล (Database Design)}
% \label{subsec:database-design}

% ฐานข้อมูลถูกออกแบบบน PostgreSQL ซึ่งเป็นฐานข้อมูลเชิงสัมพันธ์ (Relational Database) ที่มีความน่าเชื่อถือและเหมาะสมกับการจัดการข้อมูลที่มีโครงสร้างชัดเจน โครงสร้างหลัก (Tables) ที่ออกแบบไว้ประกอบด้วย:

% % มาแก้
% \begin{enumerate}[leftmargin=4em]
%     \item \textbf{users:} จัดเก็บข้อมูลของผู้ใช้
%     \item \textbf{master\_medications:} ฐานข้อมูลยาหลัก
%     \item \textbf{medications:} จัดเก็บข้อมูลยาที่ผู้ใช้ทำการบันทึกไว้
%     \item \textbf{medicatin\_reminders:} จัดเก็บตารางเวลาการแจ้งเตือนการทานยา
%     \item \textbf{symptom\_recode:} จัดเก็บประวัติการทานยาและอาการของผู้ใช้แต่ละคน
% \end{enumerate}

% โดยแต่ละตารางจะมีการกำหนด Primary Key และ Foreign Key เพื่อเชื่อมโยงข้อมูลเข้าด้วยกันอย่างเป็นระบบและป้องกันข้อมูลซ้ำซ้อน

% \clearpage
\subsection{ขั้นตอนการออกแบบระบบ}
\label{subsec:system-design-steps}

\begin{enumerate}[leftmargin=4em]

\item \textbf{Entity Relationship Diagram (ERD) :}
อธิบายโครงสร้างระหว่างฐานข้อมูลในรูปแบบแผนภาพ โดยในโครงงานนี้ จะใช้ในแนวคิด ERD แบบ Crow's Foot Notation \cite{erd} ซึ่งเป็นรูปแบบการวาด ERD ที่นิยมใช้เนื่องจากอ่านง่าย และแสดงจำนวนความสัมพันธ์ (Cardinality) ระหว่าง Entities ชัดเจน ซึ่งมีองค์ประกอบดังนี้
    \begin{enumerate}[leftmargin=4em]
    \item \textbf{Entity:} สิ่งของ บุคคล เหตุการณ์ หรือแนวคิดที่สามารถระบุได้ชัดเจน
    \item \textbf{Relationship:} ความสัมพันธ์ระหว่างเอนทิตี
    \item \textbf{Attributes:} คุณลักษณะของเอนทิตีหรือความสัมพันธ์
    \item \textbf{Primary Key:} ใช้เพื่อระบุเอนทิตีอย่างไม่ซ้ำกัน
    \item \textbf{Cardinality:} จำนวนขั้นต่ำและสูงสุดของความสัมพันธ์ระหว่างเอนทิตี
\end{enumerate}

โดยฐานข้อมูลถูกออกแบบบน PostgreSQL ซึ่งเป็นฐานข้อมูลเชิงสัมพันธ์ (Relational Database) ที่มีความน่าเชื่อถือและเหมาะสมกับการจัดการข้อมูลที่มีโครงสร้างชัดเจน โครงสร้างหลัก (Tables) ที่ออกแบบไว้ประกอบด้วย:

\begin{enumerate}[leftmargin=4em]
    \item \textbf{users:} จัดเก็บข้อมูลของผู้ใช้
    \item \textbf{master\_medications:} ฐานข้อมูลยาหลัก
    \item \textbf{medications:} จัดเก็บข้อมูลยาที่ผู้ใช้ทำการบันทึกไว้
    \item \textbf{medicatin\_reminders:} จัดเก็บตารางเวลาการแจ้งเตือนการทานยา
    \item \textbf{symptom\_recode:} จัดเก็บประวัติการทานยาและอาการของผู้ใช้แต่ละคน
\end{enumerate}
\vspace{0.6cm}
\begin{figure}[!htb]
    \centering
    \includegraphics[width=0.95\textwidth]{figures/chap4/erd/entity_relationship_diagram.png}
    \captionsetup{name=รูป}
    \caption{Entity Relationship Diagram (ERD)}
    \label{fig:erd}
\end{figure}
\clearpage

\item \textbf{Functional Decomposition:}
ขั้นตอนการแบ่งระบบออกเป็นส่วนย่อย ๆ เพื่อให้ง่ายต่อการทำความเข้าใจ การวิเคราะห์ การพัฒนา และการนำไปใช้งาน

\begin{figure}[!htb]
    \centering
    \includegraphics[width=0.95\textwidth]{figures/chap4/functional_decomposition/functional_decomposition.png}
    \captionsetup{name=รูป}
    \caption{Functional Decomposition}
    \label{fig:functional_decomposition}
\end{figure}

\item \textbf{Data Flow Diagram Level 0:}
หรือ Context Diagram ใช้แสดงภาพรวมของระบบทั้งหมด

\begin{figure}[!htb]
    \centering
    \includegraphics[width=0.9\textwidth]{figures/chap4/dfd_level_0/dfd_level_0.png}
    \captionsetup{name=รูป}
    \caption{Data Flow Diagram Level 0}
    \label{fig:dfd_level_0}
\end{figure}



\item \textbf{Data Flow Diagram Level 1 :}
ใช้แสดงกระบวนการย่อย ๆ ภายในระบบ
\begin{figure}[!htb]
    \centering
    \includegraphics[width=0.95\textwidth]{figures/chap4/dfd_level_1/dfd_level_1.png}
    \captionsetup{name=รูป}
    \caption{Data Flow Diagram Level 1}
    \label{fig:dfd_level_1}
\end{figure}
\clearpage

\item \textbf{Data Flow Diagram Level 2:}
ใช้ในการขยายข้อมูลจาก Data Flow Diagram ระดับ 1 ให้ละเอียดมากยิ่งขึ้น

\begin{enumerate}[leftmargin=4em]
\item \textbf{Manage User Information}
\begin{figure}[H]
    \centering
    \includegraphics[width=0.9\textwidth]{figures/chap4/dfd_level_2/dfd_level_2.1.png}
    \captionsetup{name=รูป}
    \caption{Data Flow Diagram Level 2.1 - Manage User Information}
    \label{fig:dfd_level_2_1}
\end{figure}

\item \textbf{OCR/ Barcode/ Text-To-Speech}
\begin{figure}[H]
    \centering
    \includegraphics[width=0.9\textwidth]{figures/chap4/dfd_level_2/dfd_level_2.2.png}
    \captionsetup{name=รูป}
    \caption{Data Flow Diagram Level 2.2 - OCR/ Barcode/ Text-To-Speech}
    \label{fig:dfd_level_2_2}
\end{figure}
\clearpage

\item \textbf{Medication Management}
\begin{figure}[H]
    \centering
    \includegraphics[width=0.9\textwidth]{figures/chap4/dfd_level_2/dfd_level_2.3.png}
    \captionsetup{name=รูป}
    \caption{Data Flow Diagram Level 2.3 - Medication Management}
    \label{fig:dfd_level_2_3}
\end{figure}

\item \textbf{Notification Management}
\begin{figure}[H]
    \centering
    \includegraphics[width=0.9\textwidth]{figures/chap4/dfd_level_2/dfd_level_2.4.png}
    \captionsetup{name=รูป}
    \caption{Data Flow Diagram Level 2.4 - Notification Management}
    \label{fig:dfd_level_2_4}
\end{figure}
\clearpage

\item \textbf{History Logging}
\begin{figure}[H]
    \centering
    \includegraphics[width=0.9\textwidth]{figures/chap4/dfd_level_2/dfd_level_2.5.png}
    \captionsetup{name=รูป}
    \caption{Data Flow Diagram Level 2.5 - History Logging}
    \label{fig:dfd_level_2_5}
\end{figure}
% \vspace{-0.5cm}
\end{enumerate}
\end{enumerate}

\subsection{การพัฒนา UX/UI:}
ในการออกแบบส่วนติดต่อผู้ใช้ (User Interface: UI) และประสบการณ์ผู้ใช้ (User Experience: UX) ของแอปพลิเคชัน ได้เลือกใช้เครื่องมือ Figma \cite{figma} ซึ่งเป็นแพลตฟอร์มออกแบบแบบออนไลน์ (Cloud-based Design Tool) ที่ได้รับความนิยมอย่างแพร่หลาย เนื่องจากรองรับการออกแบบแบบร่วมมือกัน (Collaborative Design) และสามารถแก้ไขแบบร่างได้แบบเรียลไทม์ผ่านเว็บเบราว์เซอร์โดยไม่ต้องติดตั้งโปรแกรมเพิ่มเติม

Figma ช่วยให้สามารถสร้างต้นแบบ (Prototype) ของหน้าจอแอปพลิเคชันได้อย่างสะดวก โดยรองรับการสร้างองค์ประกอบ UI เช่น ปุ่ม เมนู ตัวอักษร รวมถึงการเชื่อมโยงลำดับหน้าจอเพื่อจำลองการใช้งานจริง ทำให้ง่ายต่อการนำเสนอและทดสอบการใช้งานก่อนเริ่มพัฒนา
โดยมีหน้าแอปพลิเคชันที่ได้จากการออกแบบด้วย Figma ดังแสดงต่อไปนี้
    \begin{figure}[H]
        \centering
        \includegraphics[width=0.85\textwidth]{figures/chap4/figma/figma1.png}
        \captionsetup{name=รูป}
        \caption{หน้าการจัดการข้อมูลผู้ใช้งาน}
        \label{fig:user_management_design}
    \end{figure}
    \vspace{-0.3cm}
    \begin{figure}[H]
        \centering
        \includegraphics[width=0.85\textwidth]{figures/chap4/figma/figma2.png}
        \captionsetup{name=รูป}
        \caption{หน้าสำหรับการจัดการข้อมูลยา}
        \label{fig:medication_management_design}
    \end{figure}
    \vspace{-0.3cm}
    \begin{figure}[H]
        \centering
        \includegraphics[width=0.85\textwidth]{figures/chap4/figma/figma3.png}
        \captionsetup{name=รูป}
        \caption{หน้าสำหรับการจัดการข้อมูลอาการหลังการใช้ยา}
        \label{fig:symptom_management_design}
    \end{figure}
    \vspace{-0.3cm}
    \begin{figure}[H]
        \centering
        \includegraphics[width=0.85\textwidth]{figures/chap4/figma/figma4.png}
        \captionsetup{name=รูป}
        \caption{หน้าสำหรับการจัดการตั้งค่าการแจ้งเตือน}
        \label{fig:notification_management_design}
    \end{figure}
    \vspace{-0.3cm}
    \begin{figure}[H]
        \centering
        \includegraphics[width=0.85\textwidth]{figures/chap4/figma/figma6.png}
        \captionsetup{name=รูป}
        \caption{หน้าสำหรับการค้นหาข้อมูลยา}
        \label{fig:search_medication_design}
    \end{figure}
    \vspace{-0.3cm}
    \begin{figure}[H]
        \centering
        \includegraphics[width=0.85\textwidth]{figures/chap4/figma/figma7.png}
        \captionsetup{name=รูป}
        \caption{หน้าสำหรับการถ่ายภาพยาเพื่อยืนยันการทานยา การสแกนบาร์โค้ด และการแจ้งเตือนการทานยา}
        \label{fig:confirm_medication_design}
    \end{figure}
    \vspace{-0.3cm}
    \begin{figure}[H]
        \centering
        \includegraphics[width=0.75\textwidth]{figures/chap4/figma/figma5.png}
        \captionsetup{name=รูป}
        \caption{หน้าสำหรับการเข้าสู่ระบบ}
        \label{fig:login_design}
    \end{figure}
    
% \section{การพัฒนาและติดตั้ง (Development and Implementation)}
% \label{sec:development-implementation}


% \setlength{\baselineskip}{\normalbaselineskip}

% \subsection{ขั้นตอนการพัฒนา}
% \label{subsec:development-steps}
\section{ขั้นตอนการพัฒนา}
\label{Sec:Procedure}

การพัฒนาโครงงานนี้ใช้แนวทางการพัฒนาแบบ Agile Framework โดยแบ่งการพัฒนาออกเป็น 8 Sprint เพื่อให้สามารถปรับปรุงและพัฒนาระบบได้อย่างต่อเนื่อง โดยแต่ละ Sprint มีระยะเวลา 2 สัปดาห์ โดยแบ่งการทำงานออกเป็น 6 ขั้นตอนหลัก ดังนี้

\subsection{ขั้นตอนที่ 1: การรวบรวมปัญหา วิเคราะห์ความต้องการของผู้ใช้งานและการออกแบบ}

ในขั้นตอนนี้ได้ทำการศึกษาและวิเคราะห์ปัญหาของกลุ่มเป้าหมาย ซึ่งเป็นผู้สูงอายุที่ต้องรับประทานยาหลายชนิดต่อวัน จากนั้นรวบรวมความต้องการของผู้ใช้งาน (User Requirement) และออกแบบระบบ (System Design) ครอบคลุมการออกแบบส่วนติดต่อผู้ใช้ (UI Design), ฐานข้อมูล, การแยกฟังก์ชันการทำงานของระบบ (Functional Decomposition) และการออกแบบกระบวนการทำงาน (Data Flow Diagram: DFD)


\subsection{ขั้นตอนที่ 2: การศึกษาและพัฒนา Data Pipeline ด้วย PySpark}

ขั้นตอนนี้เป็นการออกแบบและพัฒนากระบวนการ Data Pipeline เพื่อจัดการข้อมูลยาจากแหล่งข้อมูลภายนอก ได้แก่ ข้อมูลจากระบบ Thai Medicine Terminology (TMT) จากสำนักพัฒนามาตรฐานระบบข้อมูลสุขภาพไทย  ซึ่งให้ข้อมูลยาในรูปแบบไฟล์ Excel และสำนักงานคณะกรรมการอาหารและยา โดยให้ข้อมูลยาผ่านเว็บไซต์ จากนั้นใช้ PySpark ในการประมวลผล ทำความสะอาดข้อมูล (Data Cleaning) และแปลงข้อมูลให้อยู่ในรูปแบบที่เหมาะสมก่อนนำเข้าฐานข้อมูล PostgreSQL เพื่อให้ระบบ Data Pipeline สามารถทำงานร่วมกับส่วน Backend และ Database ได้อย่างราบรื่น

\subsection{ขั้นตอนที่ 3: การจัดการ Container ด้วย Kubernetes in Docker (Kind)}

มีการใช้ Kind ในการควบคุมการทำงานของ Container ทั้งหมด โดยภายใน Cluster จะประกอบด้วย
\begin{enumerate}[leftmargin=4em]
    \item Container ของ PySpark สำหรับประมวลผลและจัดการข้อมูลยา
    \item Container ของ Spring Boot สำหรับให้บริการ API
    \item Container ของ PostgreSQL สำหรับเก็บข้อมูลที่ผ่านการประมวลผลแล้ว
\end{enumerate}

\subsection{ขั้นตอนที่ 4: พัฒนาระบบในส่วนของหน้าบ้าน (Frontend)}

ทำการออกแบบและพัฒนาแอปพลิเคชันบนระบบปฏิบัติการ Android โดยใช้ Flutter Framework ร่วมกับภาษา Dart โดยพัฒนาหน้าจอต่าง ๆ เช่น หน้าสำหรับตั้งแจ้งเตือนการทานยา หน้าสำหรับการค้นหาข้อมูลยาฯลฯ หลังจากนั้นทำการเชื่อมต่อกับระบบ API เพื่อให้สามารถดึงและบันทึกข้อมูลจากส่วน Backend ได้อย่างถูกต้อง ซึ่งแบ่งเป็น Sprint ดังนี้

\begin{enumerate}[leftmargin=4em]
    \item \textbf{Sprint1:} พัฒนาหน้าสำหรับค้นหาข้อมูลยาด้วยการสแกนบาร์โค้ด และการอ่านข้อความบนฉลากยา (OCR)
    \item \textbf{Sprint2:} พัฒนาระบบการจัดการข้อมูลยา ครอบคลุมฟังก์ชันการเพิ่ม ลบ และแก้ไขข้อมูลยา
    \item \textbf{Sprint3:} พัฒนาระบบการจัดการข้อมูลอาการหลังการใช้ยา ครอบคลุมฟังก์ชันการเพิ่ม ลบ และแก้ไขข้อมูล
    \item \textbf{Sprint4:} ทดสอบการทำงานของระบบทั้งหมด เพื่อตรวจสอบความถูกต้องและประสิทธิภาพ
    \item \textbf{Sprint5:} พัฒนาระบบการแจ้งเตือนการทานยา ครอบคลุมฟังก์ชันการตั้งเวลาและแจ้งเตือนการทานยา
    \item \textbf{Sprint7:} พัฒนาระบบการจัดการข้อมูลผู้ใช้งาน ครอบคลุมฟังก์ชันการเพิ่ม ลบ และแก้ไขข้อมูลผู้ใช้งาน
    \item \textbf{Sprint8:} ทดสอบการทำงานของระบบทั้งหมด โดยมีการทดสอบกับผู้ใช้งานจริงเพื่อรับฟังความคิดเห็น
\end{enumerate}

\subsection{ขั้นตอนที่ 5: การพัฒนาระบบในส่วนของหลังบ้าน (Backend)}

ใช้ Spring Boot Framework ในการสร้าง REST API ที่เชื่อมต่อระหว่างแอปพลิเคชันกับฐานข้อมูล PostgreSQL ซึ่งทำหน้าที่เก็บข้อมูลยา ข้อมูลผู้ใช้ ตารางการทานยา และข้อมูลการบันทึกประวัติการรับประทานยา โดยออกแบบ API endpoints ให้รองรับการทำงานต่าง ๆ เช่น การเพิ่ม แก้ไข และลบข้อมูลยา การตั้งค่าการแจ้งเตือน และการดึงข้อมูลยาจากฐานข้อมูล ซึ่งแบ่งเป็น Sprint ดังนี้
\begin{enumerate}[leftmargin=4em]
    \item \textbf{Sprint1:} พัฒนา API สำหรับค้นหาข้อมูลยา
    \item \textbf{Sprint2:} พัฒนา API สำหรับเพิ่ม ลบ และแก้ไขข้อมูลยา
    \item \textbf{Sprint3:} พัฒนา API สำหรับเพิ่ม ลบ และแก้ไขอาการหลังการรับประทานยา
    \item \textbf{Sprint4:} สร้าง Unit Test สำหรับทดสอบการทำงานของ API
    \item \textbf{Sprint5:} พัฒนา API สำหรับจัดการการแจ้งเตือนการทานยา
    \item \textbf{Sprint6:} พัฒนา API สำหรับจัดการหน้า Home Page
    \item \textbf{Sprint7:} พัฒนา API สำหรับจัดการข้อมูลผู้ใช้งาน
    \item \textbf{Sprint8:} สร้าง Unit Test สำหรับทดสอบการทำงานของ API
\end{enumerate}

\subsection{ขั้นตอนที่ 6: การทดสอบและปรับปรุงระบบ}
ในขั้นตอนนี้ทำการทดสอบระบบทั้งหมดเพื่อตรวจสอบความถูกต้องและประสิทธิภาพของแอปพลิเคชัน โดยแบ่งการทดสอบออกเป็น Unit Testing สำหรับทดสอบฟังก์ชันแต่ละส่วน Integration Testing สำหรับทดสอบการทำงานร่วมกันระหว่างส่วนต่าง ๆ และ User Acceptance Testing (UAT) โดยให้กลุ่มผู้สูงอายุทดลองใช้งานจริง จากนั้นรวบรวม Feedback และข้อเสนอแนะจากผู้ทดสอบเพื่อนำมาปรับปรุงแก้ไขระบบให้มีความสมบูรณ์และตอบสนองความต้องการของผู้ใช้งานได้ดียิ่งขึ้น


\begin{figure}[H]
    \centering
    \includegraphics[width=0.9\textwidth]{figures/chap1/planing.png}
    \captionsetup{name=รูป}
    \caption{แผนการดำเนินโครงงาน}
    \label{fig:planing}
\end{figure}

\subsection{เครื่องมือและสภาพแวดล้อมในการพัฒนา}
\label{subsec:development-tools}

\begin{enumerate}[leftmargin=4em]
    \item \textbf{ภาษาโปรแกรม:} Dart (สำหรับ Frontend), Java (สำหรับ Backend)
    \item \textbf{Framework:} Flutter, Spring Boot
    \item \textbf{Integrated Development Environment (IDE):} Android Studio และ Visual Studio Code สำหรับ Flutter, IntelliJ IDEA สำหรับ Spring Boot
    \item \textbf{จำลองบนเครื่องพัฒนา (Local Environment):} Kubernetes in Docker (kind)
    \item \textbf{ฐานข้อมูล:} PostgreSQL
\end{enumerate}
\section{การทดสอบและการประเมินผล}
\label{sec:testing-evaluation}

\subsection{การทดสอบระบบ (System Testing)}
\label{subsec:system-testing}

ทดสอบการทำงานของฟังก์ชันหลักทั้งหมด ตั้งแต่การเข้าสู่ระบบไปจนถึงการใช้งานฟีเจอร์ต่างๆ เพื่อหาข้อผิดพลาด

\subsection{การทดสอบประสิทธิภาพ (Performance Testing)}
\label{subsec:performance-testing}

วัดความเร็ว ประสิทธิภาพ และความแม่นยำของฟังก์ชันการทำงาน

\subsection{การทดสอบการยอมรับจากผู้ใช้ (User Acceptance Testing - UAT)}
\label{subsec:uat}

นำแอปพลิเคชันไปให้กลุ่มผู้สูงอายุทดลองใช้จริง เพื่อรวบรวมข้อเสนอแนะและปัญหาที่พบจากการใช้งาน เพื่อนำมาปรับปรุงเพิ่มเติม




\section{ผลลัพธ์เบื้องต้นจากการพัฒนา}
\label{subsec:initial-results}

\subsection{การทดสอบระบบเบื้องต้นในสภาพแวดล้อม Container ด้วย Kind}
\label{subsec:kind-Kubernetes}
\begin{enumerate}[leftmargin=4em]
    \item สร้าง Kind Cluster จำลองบนเครื่องพัฒนา (Local Environment) โดยมี container ดังนี้ Spring Boot Application, PySpark และ ฐานข้อมูล PostgreSQL
\end{enumerate}
\begin{figure}[H]
    \centering
    \includegraphics[width=0.9\textwidth]{figures/chap4/result/kind.png}
    \captionsetup{name=รูป}
    \caption{รูปผลลัพธ์ container ที่สร้างขึ้นจาก Kind}
    \label{fig:kind_result}
\end{figure}

\subsection{กระบวนการ Data Pipeline}
\label{subsec:data-pipeline}
\begin{enumerate}[leftmargin=4em]
    \item สามารถอ่านข้อมูลยาจากไฟล์ Excel ของ Thai Medicine Terminology (TMT) ได้
    \item PySpark สามารถประมวลผลข้อมูลที่ได้จากการอ่านไฟล์ Excel ของ Thai Medicine Terminology (TMT) ได้
    \item สามารถโหลดข้อมูลที่อ่านได้ไปยังฐานข้อมูล PostgreSQL ได้
\end{enumerate}
\begin{figure}[H]
    \centering
    \includegraphics[width=0.9\textwidth]{figures/chap4/result/db.png}
    \captionsetup{name=รูป}
    \caption{รูปผลลัพธ์จากฐานข้อมูล PostgreSQL ที่ได้จากการประมวลผลข้อมูลด้วย PySpark}
    \label{fig:database_result}
\end{figure}

\subsection{Backend}
\label{subsec:backend}
\begin{enumerate}[leftmargin=4em]
    \item สามารถเชื่อมต่อกับฐานข้อมูล PostgreSQL ได้
    \item สามารถพัฒนา API สำหรับดึงข้อมูลยาออกมาได้
\end{enumerate}
\begin{figure}[H]
    \centering
    \includegraphics[width=0.9\textwidth]{figures/chap4/result/api.png}
    \captionsetup{name=รูป}
    \caption{รูปผลลัพธ์ข้อมูลจาก API ที่ได้พัฒนาด้วย Spring Boot}
    \label{fig:api_result}
\end{figure}