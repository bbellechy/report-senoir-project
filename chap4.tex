\chapter{ระเบียบวิธีดำเนินโครงงาน}
\label{Ch:Methodology}

\section{การวิเคราะห์และออกแบบระบบ}
\label{sec:system-analysis-design}

\subsection{Persona}
\label{subsec:persona}
\cite{persona2023}
หมายถึง ตัวตนสมมติที่เป็นตัวแทนกลุ่มลูกค้าเป้าหมายของธุรกิจ โดยอิงจากข้อมูลการวิจัยและวิเคราะห์พฤติกรรม ความต้องการ และแรงจูงใจของกลุ่มลูกค้าจริง เพื่อให้ธุรกิจเข้าใจลูกค้าได้อย่างลึกซึ้ง
\subsubsection{ความสำคัญของ Persona}
\begin{enumerate}[leftmargin=4em]
    \item ช่วยให้ธุรกิจเข้าใจลูกค้าได้อย่างลึกซึ้ง
    \item ปรับกลยุทธ์ทางการตลาดให้เหมาะสมกับพฤติกรรมของกลุ่มเป้าหมาย
    \item สื่อสารได้อย่างตรงใจลูกค้า
\end{enumerate}

\begin{figure}[!htb]
    \centering
    \includegraphics[width=0.85\textwidth]{figures/chap4/persona/persona_1.png}
    \captionsetup{name=รูป}
    \caption{ตัวตนสมมติที่เป็นตัวแทนของกลุ่มผู้ใช้เป้าหมายคนที่ 1}
    \label{fig:persona_1}
\end{figure}

\begin{figure}[!htb]
    \centering
    \includegraphics[width=0.85\textwidth]{figures/chap4/persona/persona_2.png}
    \captionsetup{name=รูป}
    \caption{ตัวตนสมมติที่เป็นตัวแทนของกลุ่มผู้ใช้เป้าหมายคนที่ 2}
    \label{fig:persona_2}
\end{figure}

\begin{figure}[!htb]
    \centering
    \includegraphics[width=0.85\textwidth]{figures/chap4/persona/persona_3.png}
    \captionsetup{name=รูป}
    \caption{ตัวตนสมมติที่เป็นตัวแทนของกลุ่มผู้ใช้เป้าหมายคนที่ 3}
    \label{fig:persona_3}
\end{figure}

\clearpage


\vspace*{-2cm}

\subsection{User Journey Map}
\label{subsec:user-journey-map}
\vspace*{0pt}
\cite{journey}
หรือ แผนภาพเส้นทางผู้ใช้ เป็นเครื่องมือที่ใช้อธิบายขั้นตอนและประสบการณ์ของผู้ใช้เมื่อมีปฏิสัมพันธ์กับแอปพลิเคชันของเรา ว่าผู้ใช้โต้ตอบอย่างไร พบปัญหาตรงไหน และจะปรับปรุงจุดใดได้บ้าง เพื่อให้ผู้ใช้ได้รับประสบการณ์ที่ดีขึ้น ซึ่งจะแสดงภาพรวมตั้งแต่จุดเริ่มต้นที่ผู้ใช้เริ่มสนใจไปจนถึงการบรรลุเป้าหมายโดยแบ่งเป็นช่วงเวลาและเหตุการณ์สำคัญต่าง ๆ
%  โดยมีองค์ประกอบ ดังนี้

% \begin{enumerate}[leftmargin=2em]
%     \item \textbf{The User:} ผู้ใช้เป้าหมาย แรงจูงใจ และปัญหาที่พบ
%     \item \textbf{The Scenario \& Objective:} สถานการณ์และวัตถุประสงค์เพื่ออธิบายว่าผู้ใช้กำลังพยายามทำอะไร
%     \item \textbf{Journey Phases:} ช่วงเส้นทางของผู้ใช้ โดยจะแบ่งออกเป็น 5 ช่วงหลัก ได้แก่
%     \begin{enumerate}
%         \item \textbf{Awareness:} เริ่มรู้จักผลิตภัณฑ์
%         \item \textbf{Consideration:} พิจารณา
%         \item \textbf{Decision:} ตัดสินใจ
%         \item \textbf{Purchase:} ซื้อหรือใช้งาน
%         \item \textbf{Retention:} กลับมาใช้อีก
%     \end{enumerate}
%     \item \textbf{Actions, Attitudes, and Emotions:} พฤติกรรมและอารมณ์ของผู้ใช้ในแต่ละช่วง
%     \item \textbf{Opportunities:} โอกาสในการปรับปรุง วิเคราะห์ข้อมูลที่ได้เพื่อหาทางปรับปรุง
% \end{enumerate}

\begin{figure}[!htb]
    \centering
    \includegraphics[width=0.95\textwidth]{figures/chap4/user_journey/journey_1.png}
    \captionsetup{name=รูป}
    \caption{ประสบการณ์ของผู้ใช้เมื่อมีปฏิสัมพันธ์กับแอปพลิเคชันของเราคนที่ 1}
    \label{fig:journey_1}
\end{figure}

\begin{figure}[!htb]
    \centering
    \includegraphics[width=0.95\textwidth]{figures/chap4/user_journey/journey_2.png}
    \captionsetup{name=รูป}
    \caption{ประสบการณ์ของผู้ใช้เมื่อมีปฏิสัมพันธ์กับแอปพลิเคชันของเราคนที่ 2}
    \label{fig:journey_2}
\end{figure}

\begin{figure}[!htb]
    \centering
    \includegraphics[width=0.95\textwidth]{figures/chap4/user_journey/journey_3.png}
    \captionsetup{name=รูป}
    \caption{ประสบการณ์ของผู้ใช้เมื่อมีปฏิสัมพันธ์กับแอปพลิเคชันของเราคนที่ 3}
    \label{fig:journey_3}
\end{figure}

\subsection{ปัญหาและแนวทางการแก้ไข}
\label{subsec:problems-solutions}

ผู้สูงอายุมักลืมทานยาหรือสับสนเรื่องชนิดและเวลาทานยา ปัญหานี้แก้ได้ด้วย:

\begin{enumerate}[leftmargin=4em]
    \item ระบบแจ้งเตือนที่ตั้งเวลาได้ตามแพทย์กำหนด
    \item ฐานข้อมูลยาที่ครบถ้วน เชื่อถือได้ และเข้าใจง่าย
    \item UI ที่ชัดเจนและใช้งานง่ายสำหรับผู้สูงอายุ
    \item ฟังก์ชัน Text-to-Speech และการแจ้งเตือนด้วยเสียง/สั่น/แสง
    \item ความปลอดภัยของข้อมูลสุขภาพและการเชื่อมต่อกับอุปกรณ์สุขภาพอื่น ๆ
\end{enumerate}

\subsection{การวิเคราะห์ความต้องการ (Requirement Analysis)}
\label{subsec:requirement-analysis}

\subsubsection{ความต้องการของผู้ใช้ (User Requirements)}

\begin{enumerate}[leftmargin=4em]
    \item ระบบจัดการข้อมูลยา (เพิ่ม/แก้ไข/ลบ)
    \item ระบบแจ้งเตือนการทานยาที่ตั้งเวลาได้
    \item ระบบการสแกนข้อมูลยาจากฉลาก (Barcode \& OCR)
    \item ระบบบันทึกประวัติการทานยาและอาการ
    \item ระบบอำนวยความสะดวกสำหรับผู้สูงอายุ (Text-to-Speech)
\end{enumerate}

\subsubsection{ความต้องการของระบบ (System Requirements)}

\begin{enumerate}[leftmargin=4em]
    \item สามารถทำงานได้บน Android
    \item มีความเร็วและความแม่นยำในการสแกนข้อมูล
    \item มีการจัดเก็บข้อมูลผู้ใช้และข้อมูลยาอย่างปลอดภัย
\end{enumerate}

\subsection{การออกแบบสถาปัตยกรรม (System Architecture Design)}
\label{subsec:system-architecture}

\begin{enumerate}[leftmargin=2em]
\item \textbf{Source:}
ข้อมูลต้นทางมาจาก สำนักพัฒนามาตรฐานระบบข้อมูลสุขภาพไทย ซึ่งเผยแพร่ข้อมูลในรูปแบบไฟล์ CSV ข้อมูลดังกล่าวจะถูกนำมาใช้เป็นแหล่งข้อมูลหลักของระบบ

\item \textbf{Data Processing:}
ข้อมูลจากไฟล์ CSV จะถูกนำเข้าสู่กระบวนการประมวลผลด้วย PySpark ซึ่งเป็นเครื่องมือที่เหมาะสำหรับการจัดการและประมวลผลข้อมูลขนาดใหญ่ (Big Data Processing) เพื่อจัดรูปแบบข้อมูลให้อยู่ในโครงสร้างที่เหมาะสมต่อการจัดเก็บและการนำไปใช้งานต่อไป

\item \textbf{Database:}
หลังจากผ่านการประมวลผลแล้ว ข้อมูลจะถูกโหลดเข้าสู่ฐานข้อมูล PostgreSQL เพื่อจัดเก็บอย่างเป็นระบบและสามารถเข้าถึงได้อย่างมีประสิทธิภาพ

\item \textbf{Backend Service:}
ใช้ Spring Boot Framework ในการพัฒนาเป็นบริการฝั่งเซิร์ฟเวอร์ (Backend Service) ทำหน้าที่เป็นตัวกลางระหว่างฐานข้อมูลและฝั่งผู้ใช้งาน โดยจัดการคำขอ (Request) จากฝั่ง Client รวมถึงประมวลผลและส่งข้อมูลกลับไปยังผู้ใช้งานในรูปแบบ API

\item \textbf{Client Application:}
ส่วนของผู้ใช้งาน (Frontend) ถูกพัฒนาด้วย Dart และ Flutter ซึ่งเป็นเครื่องมือสำหรับพัฒนาแอปพลิเคชันแบบ Cross-platform เพื่อให้สามารถใช้งานได้ทั้งบนระบบปฏิบัติการ Android, iOS และ Web

\item \textbf{System Deployment (kind Cluster):}
สำหรับการทดสอบและจำลองสภาพแวดล้อมการทำงานของระบบบน Kubernetes ได้มีการใช้เครื่องมือ kind (Kubernetes IN Docker) เพื่อสร้าง Kubernetes cluster จำลองบนเครื่องพัฒนา (Local Environment) โดย cluster ที่สร้างขึ้นจะประกอบด้วย container หลัก ได้แก่ Spring Boot Application Container และ PostgreSQL Database Container

% ซึ่งเชื่อมต่อกันผ่าน Kubernetes service ภายใน cluster

การใช้ kind ช่วยให้สามารถจำลองขั้นตอนการ deploy ระบบจริง เช่น การสร้าง Pod, Service, และการ expose port ให้สามารถเข้าถึงได้จากภายนอก เหมือนการใช้งานบนสภาพแวดล้อม Cloud จริง ทั้งยังช่วยลดค่าใช้จ่ายและเวลาในการเตรียมระบบทดสอบอีกด้วย

\begin{figure}[H]
    \centering
    \includegraphics[width=0.9\textwidth]{figures/chap4/architecture/architecture.png}
    \captionsetup{name=รูป}
    \caption{System Architecture Design}
    \label{fig:system_architecture}
\end{figure}
\end{enumerate}



\subsection{การออกแบบฐานข้อมูล (Database Design)}
\label{subsec:database-design}

ฐานข้อมูลถูกออกแบบบน PostgreSQL ซึ่งเป็นฐานข้อมูลเชิงสัมพันธ์ (Relational Database) ที่มีความน่าเชื่อถือและเหมาะสมกับการจัดการข้อมูลที่มีโครงสร้างชัดเจน โครงสร้างหลัก (Tables) ที่ออกแบบไว้ประกอบด้วย:

\begin{enumerate}[leftmargin=4em]
    \item \textbf{users:} จัดเก็บข้อมูลของผู้ใช้
    \item \textbf{master\_drugs:} ฐานข้อมูลยาหลัก
    \item \textbf{medications:} จัดเก็บข้อมูลยาที่ผู้ใช้ทำการบันทึกไว้
    \item \textbf{schedules:} จัดเก็บตารางเวลาการแจ้งเตือนการทานยา
    \item \textbf{histories:} จัดเก็บประวัติการทานยาและอาการของผู้ใช้แต่ละคน
\end{enumerate}

โดยแต่ละตารางจะมีการกำหนด Primary Key และ Foreign Key เพื่อเชื่อมโยงข้อมูลเข้าด้วยกันอย่างเป็นระบบและป้องกันข้อมูลซ้ำซ้อน

\subsection{ขั้นตอนการออกแบบระบบ}
\label{subsec:system-design-steps}

\begin{enumerate}[leftmargin=2em]
\item \textbf{การพัฒนา UX/UI:}
ออกแบบและสร้างหน้าจอการใช้งานตามหลัก Human-Computer Interaction (HCI) ที่เหมาะสมกับผู้สูงอายุ เช่น ใช้
ขนาดตัวอักษรที่ใหญ่, สีสันที่ตัดกันชัดเจน และมีปุ่มกดที่ชัดเจน โดยใช้ \cite{figma} Figma เป็นเครื่องมือออกแบบ UI/UX แบบออนไลน์ (Cloud-based) 

\begin{figure}[H]
    \centering
    \includegraphics[width=0.85\textwidth]{figures/chap4/figma/figma.png}
    \captionsetup{name=รูป}
    \caption{ออกแบบ UX/UI ของ Application}
    \label{fig:ui_design}
\end{figure}
\clearpage
\item \textbf{Entity Relationship Diagram (ERD) \cite{erd} :}
อธิบายโครงสร้างระหว่างฐานข้อมูลในรูปแบบแผนภาพ โดยในโครงงานนี้ จะใช้ในแนวคิด ERD แบบ Crow's Foot Notation ซึ่งจะประกอบไปด้วย
\begin{enumerate}[leftmargin=2em]
    \item \textbf{Entity:} สิ่งของ บุคคล เหตุการณ์ หรือแนวคิดที่สามารถระบุได้ชัดเจน
    \item \textbf{Relationship:} ความสัมพันธ์ระหว่างเอนทิตี
    \item \textbf{Attributes:} คุณลักษณะของเอนทิตีหรือความสัมพันธ์
    \item \textbf{Primary Key:} ใช้เพื่อระบุเอนทิตีอย่างไม่ซ้ำกัน
    \item \textbf{Cardinality:} จำนวนขั้นต่ำและสูงสุดของความสัมพันธ์ระหว่างเอนทิตี เช่น คุณครู 1 คน มีสอนได้หลายวิชา
\end{enumerate}

% \textbf{สัญลักษณ์:}
% \begin{itemize}[leftmargin=2em]
%     \item \textbf{$\circ$ (วงกลม):} Zero ซึ่งสามารถไม่มีค่าได้
%     \item \textbf{< (ขาแฉกสามง่าม):} Many หมายถึงมีได้หลายรายการ
%     \item \textbf{$\circ$<:} Zero or Many หมายถึงอาจไม่มีค่าได้หรือมีได้หลายรายการ
%     \item \textbf{1 (ขาแฉกเดียว):} One หมายถึงมีได้หนึ่งรายการเท่านั้น
%     \item \textbf{$\circ$1:} Zero or One หมายถึงอาจไม่มีค่าได้หรือมีได้หนึ่งรายการเท่านั้น
% \end{itemize}


\begin{figure}[!htb]
    \centering
    \includegraphics[width=0.95\textwidth]{figures/chap4/erd/entity_relationship_diagram.png}
    \captionsetup{name=รูป}
    \caption{Entity Relationship Diagram (ERD)}
    \label{fig:erd}
\end{figure}
\clearpage

\item \textbf{Functional Decomposition:}
ขั้นตอนการแบ่งระบบออกเป็นส่วนย่อย ๆ เพื่อให้ง่ายต่อการทำความเข้าใจ การวิเคราะห์ การพัฒนา และการนำไปใช้งาน

\begin{figure}[!htb]
    \centering
    \includegraphics[width=0.95\textwidth]{figures/chap4/functional_decomposition/functional_decomposition.png}
    \captionsetup{name=รูป}
    \caption{Functional Decomposition}
    \label{fig:functional_decomposition}
\end{figure}

\item \textbf{Data Flow Diagram Level 0:}
หรือ Context Diagram ใช้แสดงภาพรวมของระบบทั้งหมด

\begin{figure}[!htb]
    \centering
    \includegraphics[width=0.9\textwidth]{figures/chap4/dfd_level_0/dfd_level_0.png}
    \captionsetup{name=รูป}
    \caption{Data Flow Diagram Level 0}
    \label{fig:dfd_level_0}
\end{figure}



\item \textbf{Data Flow Diagram Level 1 \cite{dfd2025} :}
ใช้แสดงกระบวนการย่อย ๆ ภายในระบบ
\begin{figure}[!htb]
    \centering
    \includegraphics[width=0.95\textwidth]{figures/chap4/dfd_level_1/dfd_level_1.png}
    \captionsetup{name=รูป}
    \caption{Data Flow Diagram Level 1}
    \label{fig:dfd_level_1}
\end{figure}
\clearpage

\item \textbf{Data Flow Diagram Level 2 \cite{dfd2025} :}
ใช้ในการขยายข้อมูลจาก Data Flow Diagram ระดับ 1 ให้ละเอียดมากยิ่งขึ้น

\begin{enumerate}[leftmargin=2em]
\item \textbf{Manage User Information}
\begin{figure}[H]
    \centering
    \includegraphics[width=0.9\textwidth]{figures/chap4/dfd_level_2/dfd_level_2.1.png}
    \captionsetup{name=รูป}
    \caption{Data Flow Diagram Level 2.1 - Manage User Information}
    \label{fig:dfd_level_2_1}
\end{figure}

\item \textbf{OCR/ Barcode/ Text-To-Speech}
\begin{figure}[H]
    \centering
    \includegraphics[width=0.9\textwidth]{figures/chap4/dfd_level_2/dfd_level_2.2.png}
    \captionsetup{name=รูป}
    \caption{Data Flow Diagram Level 2.2 - OCR/ Barcode/ Text-To-Speech}
    \label{fig:dfd_level_2_2}
\end{figure}
\clearpage

\item \textbf{Medication Management}
\begin{figure}[H]
    \centering
    \includegraphics[width=0.9\textwidth]{figures/chap4/dfd_level_2/dfd_level_2.3.png}
    \captionsetup{name=รูป}
    \caption{Data Flow Diagram Level 2.3 - Medication Management}
    \label{fig:dfd_level_2_3}
\end{figure}

\item \textbf{Notification Management}
\begin{figure}[H]
    \centering
    \includegraphics[width=0.9\textwidth]{figures/chap4/dfd_level_2/dfd_level_2.4.png}
    \captionsetup{name=รูป}
    \caption{Data Flow Diagram Level 2.4 - Notification Management}
    \label{fig:dfd_level_2_4}
\end{figure}
\clearpage

\item \textbf{History Logging}
\begin{figure}[H]
    \centering
    \includegraphics[width=0.9\textwidth]{figures/chap4/dfd_level_2/dfd_level_2.5.png}
    \captionsetup{name=รูป}
    \caption{Data Flow Diagram Level 2.5 - History Logging}
    \label{fig:dfd_level_2_5}
\end{figure}
\end{enumerate}
\end{enumerate}



\section{การพัฒนาและติดตั้ง (Development and Implementation)}
\label{sec:development-implementation}

ขั้นตอนนี้จะเปลี่ยนการออกแบบให้กลายเป็นแอปพลิเคชันที่สามารถใช้งานได้จริง

\subsection{เครื่องมือและสภาพแวดล้อมในการพัฒนา}
\label{subsec:development-tools}

\begin{enumerate}[leftmargin=2em]
    \item \textbf{ภาษาโปรแกรม:} Dart (สำหรับ Frontend), Java (สำหรับ Backend)
    \item \textbf{Framework:} Flutter, Spring Boot
    \item \textbf{IDE (Integrated Development Environment):} Android Studio และ Visual Studio Code สำหรับ Flutter, IntelliJ IDEA สำหรับ Spring Boot
    \item \textbf{Kubernetes in Docker (kind):} จำลองบนเครื่องพัฒนา (Local Environment)
    \item \textbf{ฐานข้อมูล:} PostgreSQL
    \item \textbf{Plugins/Libraries ที่ใช้:}
    \begin{itemize}[leftmargin=2em]
        \item \texttt{flutter\_local\_notifications:} สำหรับระบบแจ้งเตือน
        \item \texttt{ google\_mlkit\_text\_recognition:} สำหรับฟังก์ชัน OCR
        \item \texttt{mobile\_scanner:} สำหรับการสแกน Barcode
        \item \texttt{flutter\_tts:} สำหรับฟังก์ชัน Text-to-Speech
    \end{itemize}
\end{enumerate}

\subsection{ขั้นตอนการพัฒนา}
\label{subsec:development-steps}

\subsubsection{การพัฒนาในส่วนของหน้าบ้าน (Frontend Development)}
พัฒนาในส่วนของ แอปพลิเคชันบนระบบ Android โดยใช้ Flutter (ภาษา Dart) ซึ่งเป็น Framework ที่ช่วยให้การพัฒนา Mobile Application ภายในแอปพลิเคชันมีฟังก์ชันหลักดังนี้

\begin{enumerate}[leftmargin=2em]
    \item สแกนฉลากยา (OCR) ด้วยไลบรารี flutter_tesseract_ocr เพื่ออ่านข้อความจากฉลากยา
    \item สแกนบาร์โค้ด ด้วย mobile_scanner เพื่อค้นหาข้อมูลยา
    \item Text-to-Speech (flutter\_tts) เพื่ออ่านข้อความยาให้ผู้ใช้งานฟัง
    \item flutter\_local\_notifications สำหรับตั้งเวลาและแจ้งเตือนการรับประทานยา
\end{enumerate}

ระบบฝั่ง Frontend จะทำการเชื่อมต่อกับ API ของ Backend ผ่าน HTTP Request เพื่อดึงข้อมูลยา เพิ่มข้อมูลการใช้งาน และบันทึกอาการหลังรับประทานยา

\subsubsection{การพัฒนาในส่วนของหลังบ้าน (Backend Development)}
ส่วนของหลังบ้านพัฒนาด้วย Spring Boot Framework ซึ่งทำหน้าที่เป็นตัวกลางในการเชื่อมต่อระหว่างแอปพลิเคชันกับฐานข้อมูล PostgreSQL โดยมีการออกแบบ RESTful API สำหรับให้บริการข้อมูล เช่น
\begin{enumerate}[leftmargin=2em]
    \item API สำหรับค้นหาข้อมูลยา
    \item API สำหรับเพิ่ม ลบ และแก้ไขข้อมูลยา
    \item API สำหรับเพิ่ม ลบ และแก้ไขอาการหลังการรับประทานยา
    \item API สำหรับจัดการข้อมูลผู้ใช้งาน
\end{enumerate}

\subsubsection{การสร้างและประมวลผลข้อมูลยา (Data Pipeline Development)}
ได้ทำการสร้าง Data Pipeline เพื่อจัดการข้อมูลยา โดยใช้เครื่องมือ PySpark สำหรับประมวลผลข้อมูลขนาดใหญ่จากแหล่งข้อมูลภายนอก

ข้อมูลยาหลักที่ใช้คือจาก Thai Medicine Terminology (TMT) ซึ่งให้ข้อมูลในรูปแบบไฟล์ .xlx โดยมีการดำเนินการดังนี้
\begin{enumerate}[leftmargin=2em]
    \item นำเข้าข้อมูล (Data Ingestion) จากไฟล์ TMT
    \item ทำความสะอาดข้อมูล (Data Cleaning) จัดรูปแบบข้อความให้เป็นมาตรฐาน
    \item แปลงรูปแบบข้อมูล (Data Transformation) ให้อยู่ในโครงสร้างที่เหมาะสมกับฐานข้อมูล
    \item ตรวจสอบความถูกต้อง (Validation) เพื่อให้แน่ใจว่าข้อมูลสมบูรณ์
    \item บันทึกข้อมูล (Data Loading) ลงในฐานข้อมูล PostgreSQL
\end{enumerate}
\subsubsection{การจำลองและติดตั้งระบบด้วย Kind (Kubernetes in Docker)}
เพื่อให้สามารถทดสอบระบบในสภาพแวดล้อมจริงแบบ Containerized ได้อย่างมีประสิทธิภาพ ได้มีการใช้เครื่องมือ Kind (Kubernetes in Docker) ซึ่งเป็นแพลตฟอร์มสำหรับจำลองการทำงานของ Kubernetes บนเครื่องพัฒนา โดยมีโครงสร้างดังนี้
\begin{enumerate}[leftmargin=2em]
    \item Container ของ PySpark สำหรับประมวลผลข้อมูลยา
    \item Container ของ Spring Boot (Backend) สำหรับให้บริการ API
    \item Container ของ PostgreSQL (Database) สำหรับเก็บข้อมูลยาและข้อมูลผู้ใช้งาน
\end{enumerate}
\section{การทดสอบและการประเมินผล}
\label{sec:testing-evaluation}

การทดสอบมีความสำคัญเพื่อรับประกันว่าแอปพลิเคชันทำงานได้อย่างถูกต้องและมีคุณภาพ

\subsection{การทดสอบระบบ (System Testing)}
\label{subsec:system-testing}

ทดสอบการทำงานของฟังก์ชันหลักทั้งหมด ตั้งแต่การเข้าสู่ระบบไปจนถึงการใช้งานฟีเจอร์ต่างๆ เพื่อหาข้อผิดพลาด

\subsection{การทดสอบประสิทธิภาพ (Performance Testing)}
\label{subsec:performance-testing}

วัดความเร็วและความแม่นยำของฟังก์ชันสำคัญ เช่น ความเร็วในการสแกน Barcode และ OCR

\subsection{การทดสอบการยอมรับจากผู้ใช้ (User Acceptance Testing - UAT)}
\label{subsec:uat}

นำแอปพลิเคชันไปให้กลุ่มผู้สูงอายุทดลองใช้จริง เพื่อรวบรวมข้อเสนอแนะและปัญหาที่พบจากการใช้งาน ซึ่งจะนำไปสู่การปรับปรุงในเวอร์ชันต่อไป
