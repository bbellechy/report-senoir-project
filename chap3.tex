\chapter{ผลงานที่เกี่ยวข้อง}
\label{Ch:RelatedWork}

\section{แอปพลิเคชันช่วยอ่านฉลากยาโดยเฉพาะฟังก์ชันที่ช่วยให้อ่านข้อมูลยาได้ง่ายขึ้นอย่างชัดเจน}
\label{sec:related-work-1}
\textbf{ผู้วิจัย:} ณัฐกรณ์ ศรีบุรมย์ \cite{thaijournal2023}

\textbf{ที่มา:} ผู้สูงอายุที่มีอายุมากกว่า 60 ปีขึ้นไป มีปัญหาด้านสายตา ความจำ และไม่คุ้นเคยกับเทคโนโลยี ซึ่งส่งผลให้เกิดความยากลำบากในการอ่านฉลากยาและปฏิบัติตามคำแนะนำในการใช้ยาได้อย่างถูกต้อง

\textbf{วิธีการ:} พัฒนาแอปพลิเคชันที่เป็นตัวช่วยในการอ่านฉลากยา โดยใช้เทคโนโลยีการรู้จำอักขระด้วยแสง (OCR) เพื่อแปลงข้อความบนฉลากยาให้เป็นข้อมูลดิจิทัล และนำเสนอข้อมูลในรูปแบบที่เข้าใจง่าย มีตัวอักษรขนาดใหญ่ เพื่อส่งเสริมพฤติกรรมการใช้ยาที่ถูกต้อง

\textbf{ผลลัพธ์:} จากการทดสอบกับกลุ่มอาสาสมัคร พบว่าอาสาสมัครสามารถตอบคำถามเกี่ยวกับวิธีการรับประทานยา จำนวนครั้งต่อวัน จำนวนเม็ด และต้องใช้ยาก่อนหรือหลังมื้ออาหารหรือเวลามีอาการ ได้ถูกต้องทั้งหมดทุกข้อและในทุกฉลาก แสดงให้เห็นว่าแอปพลิเคชันสามารถช่วยให้ผู้สูงอายุเข้าใจข้อมูลยาได้ดีขึ้น

\textbf{ข้อดี:}
\begin{enumerate}[leftmargin=4em]
    \item เป็นแอปพลิเคชันที่ตอบโจทย์ผู้สูงอายุได้อย่างตรงจุด
    \item ช่วยลดความผิดพลาดในการใช้ยาอย่างมีนัยสำคัญ
    \item ใช้งานง่าย เหมาะสมกับผู้สูงอายุที่ไม่คุ้นเคยกับเทคโนโลยี
\end{enumerate}

\textbf{ข้อจำกัด:}
\begin{enumerate}[leftmargin=4em]
    \item การสแกนข้อมูลบางครั้งมีความหน่วง (delay) เพื่อให้การอ่านฉลากยาในครั้งก่อนหน้าเสร็จเรียบร้อยก่อนดำเนินการสแกนครั้งต่อไป
    \item อาจมีข้อจำกัดในการรองรับฉลากยาที่มีรูปแบบแตกต่างกัน
\end{enumerate}

\section{แอปพลิเคชันบน LINE LIFF ร่วมกับ Google Apps Script สำหรับติดตามและประเมินผลความสม่ำเสมอในการรับประทานยาของผู้ป่วยวัณโรค โรงพยาบาลชัยนาทนเรนทร}
\label{sec:related-work-2}
\textbf{ผู้วิจัย:} วิศวัสต์ ปาริยะประเสริฐ \cite{linelife2025}

\textbf{ที่มา:} ปัญหาของผู้ป่วยที่ขาดความสม่ำเสมอในการทานยา โดยเฉพาะในกลุ่มโรคเรื้อรังเช่นวัณโรค และผู้สูงอายุที่มักจะลืมหรือสับสนเรื่องการทานยา ซึ่งส่งผลกระทบต่อประสิทธิภาพการรักษาและอาจทำให้เกิดภาวะแทรกซ้อนได้

\textbf{วิธีการ:} ใช้แอปพลิเคชันขนาดเล็กบนแพลตฟอร์ม LINE LIFF (LINE Front-end Framework) เพื่อติดตามการทานยาของผู้ป่วยวัณโรค โดยแอปพลิเคชันจะทำหน้าที่ดังนี้
\begin{enumerate}[leftmargin=4em]
    \item ส่งข้อความแจ้งเตือนผู้ป่วยเมื่อถึงเวลาทานยา
    \item ให้ผู้ป่วยส่งภาพถ่ายยืนยันการทานยาแต่ละครั้ง
    \item บันทึกตำแหน่งที่ผู้ใช้ทานยาได้
    \item จัดเก็บข้อมูลลงในฐานข้อมูลเพื่อการติดตามผลและวิเคราะห์
\end{enumerate}

\textbf{ผลลัพธ์:} หลังจากผู้ป่วยใช้แอปพลิเคชันเป็นเวลา 4 เดือน พบว่าค่าเฉลี่ยความสม่ำเสมอในการทานยาเพิ่มขึ้นจาก 90.66\% เป็น 98.38\% อย่างมีนัยสำคัญทางสถิติ ซึ่งแสดงให้เห็นว่าการแจ้งเตือนและการติดตามผลอย่างเป็นระบบช่วยให้ผู้ป่วยมีวินัยในการทานยามากขึ้น และสามารถลดโอกาสการลืมทานยาได้อย่างมีประสิทธิภาพ

\textbf{ข้อดี:}
\begin{enumerate}[leftmargin=4em]
    \item ใช้แพลตฟอร์ม LINE ซึ่งเป็นแอปพลิเคชันที่ผู้ป่วยคุ้นเคยและใช้งานอยู่แล้ว ทำให้เรียนรู้และเข้าถึงได้ง่าย
    \item ต้นทุนในการพัฒนาระบบไม่สูงมาก
    \item ผลลัพธ์ที่ได้มีความชัดเจนและวัดผลได้
    \item สามารถติดตามและวิเคราะห์พฤติกรรมการทานยาได้อย่างเป็นระบบ
\end{enumerate}

\textbf{ข้อเสนอแนะ:}
\begin{enumerate}[leftmargin=4em]
    \item เนื่องจากจะมีการยุติการให้บริการ LINE Notify ตั้งแต่วันที่ 31 มีนาคม พ.ศ. 2568 เป็นต้นไป จึงจำเป็นต้องใช้โปรแกรมอื่นมาใช้ทดแทน LINE Notify เช่น Google Chat หรือ Telegram สำหรับการแจ้งเตือนต่อไป
    \item ควรมีการพัฒนาระบบให้รองรับการทำงานแบบ standalone มากขึ้น เพื่อลดการพึ่งพาแพลตฟอร์มเฉพาะ
\end{enumerate}

\section{โครงการวิจัยเรื่อง เทคโนโลยีเภสัชสารสนเทศแสดงข้อมูลฉลากยาเอกสารกํากับยาแบบอัตโนมัติสําหรับ บริหารจัดการคลังยาปฎิชีวนะเพื่อความปลอดภัย}
\label{sec:related-work-4}
\textbf{ผู้วิจัย:} วิรุฬห์ ศรีบริรักษ์ \cite{automicpharmacy2018}

\textbf{ที่มา:} ผู้บริโภคขาดความรู้ความเข้าใจถึงอันตรายที่เกี่ยวกับยา อีกทั้งยังไม่มีแหล่งข้อมูลที่เข้าถึงได้ง่าย ข้อมูลบนฉลากยาจากคลินิกหรือโรงพยาบาลมักไม่เพียงพอต่อการทำความเข้าใจ นอกจากนี้ยังมีปัญหาในการบริหารจัดการคลังยาที่ต้องอาศัยการบันทึกข้อมูลด้วยมือซึ่งใช้เวลานานและเสี่ยงต่อความผิดพลาด

\clearpage
\textbf{วิธีการ:} พัฒนาระบบที่ใช้เทคโนโลยี Optical Character Recognition (OCR) ในการแปลงไฟล์ภาพเอกสารที่ได้รับการสแกนให้กลายเป็นไฟล์ข้อความดิจิทัล และจัดเก็บในฐานข้อมูลในเครื่องแม่ข่าย ระบบนี้ทำหน้าที่เป็นเครื่องมือที่ช่วยในการ
\begin{enumerate}[leftmargin=4em]
    \item ลดความคลาดเคลื่อนทางยา (Medication Error)
    \item ป้องกันความผิดพลาดในการจัดการคลังยา
    \item ให้ข้อมูลยาที่ถูกต้องและครบถ้วนแก่ผู้ใช้
\end{enumerate}

\textbf{ผลลัพธ์:} ระบบที่พัฒนาขึ้นมีความแม่นยำในการแปลงข้อมูลสูงถึงร้อยละ 96.61 และสามารถช่วยลดเวลาของการเก็บข้อมูลลงฐานข้อมูลยาได้อย่างมีนัยสำคัญ เมื่อเทียบกับการบันทึกข้อมูลด้วยมือ

\textbf{ข้อดี:}
\begin{enumerate}[leftmargin=4em]
    \item ช่วยให้ผู้บริโภคมีความเข้าใจในการใช้ยามากขึ้น
    \item ช่วยให้การใช้ยามีความปลอดภัยและลดปัญหาของการใช้ยาที่ผิดกับโรค
    \item ลดเวลาและแรงงานในการบันทึกข้อมูลยา
    \item มีความแม่นยำสูงในการอ่านข้อมูล
\end{enumerate}

\textbf{ข้อจำกัด:}
\begin{enumerate}[leftmargin=4em]
    \item ยังคงต้องพัฒนาความยืดหยุ่นของระบบให้รองรับรูปแบบเอกสารที่แตกต่างกัน เช่น ฉลากยาจากโรงพยาบาลต่างๆ ที่อาจมีรูปแบบไม่เหมือนกัน
    \item ความแม่นยำอาจลดลงเมื่อฉลากยามีคุณภาพภาพต่ำหรือข้อมูลไม่ชัดเจน
    \item ต้องมีการปรับปรุงและฝึกฝนโมเดล OCR อย่างต่อเนื่องเพื่อรองรับฟอนต์และรูปแบบตัวอักษรที่หลากหลาย
\end{enumerate}

\section{การเปรียบเทียบแอปพลิเคชันที่มีอยู่ในตลาด}
\label{sec:market-comparison}

การวิเคราะห์แอปพลิเคชันที่อยู่ในตลาดช่วยให้เห็นภาพรวมของฟีเจอร์และแนวทางการออกแบบที่ได้รับความนิยม ซึ่งสามารถนำมาประยุกต์ใช้ในการพัฒนาแอปพลิเคชันได้

โดยได้ทำการเปรียบเทียบฟีเจอร์หลักของแอปพลิเคชันยอดนิยม 5 ตัว ได้แก่ CapYaDoo, PharmaSee \cite{pharmasee}, MyYaAndYou \cite{myyaandyou}, Medisafe \cite{medisafe} และ RDU รู้เรื่องยา \cite{rdu} ดังตารางต่อไปนี้

\subsection{ตารางเปรียบเทียบแอปพลิเคชันในตลาด}

\begin{table}[H]
\centering
\resizebox{\textwidth}{!}{%
\begin{tabular}{|p{2.8cm}|p{2.2cm}|p{2.2cm}|p{2.2cm}|p{2.2cm}|p{2.2cm}|}
\hline
\textbf{หมวดฟีเจอร์} & \textbf{CapYaDoo} & \textbf{PharmaSee} & \textbf{MyYaAndYou} & \textbf{Medisafe} & \textbf{RDU รู้เรื่องยา} \\ 
\hline
ค้นหาข้อมูลยา & ค้นหาด้วยข้อความ / สแกน / แปลภาษาได้ & ค้นหาจากภาพยา (AI ตรวจรูป) & ค้นหาด้วยชื่อยา & ค้นหาด้วยชื่อยาเท่านั้น & ค้นหาด้วยข้อความ / QR Code \\ 
\hline
เสียงอ่านข้อมูลยา (Text-to-Speech) & มีเสียงอ่านข้อมูลยาให้ผู้ใช้ฟัง & ไม่มี & ไม่มี & ไม่มี & มีเสียงอ่านข้อมูลยาให้ผู้ใช้ฟัง \\ 
\hline
บันทึกอาการหลัง/การแพ้ & มีระบบบันทึกอาการและผลข้างเคียง & ไม่มี & มีบันทึกยาและบันทึกอาการ & ไม่มี & ไม่มี \\ 
\hline
แจ้งเตือนการทานยา & ตั้งเวลาเตือนตามวันเวลาที่ต้องการได้ & ไม่มี & มีระบบแจ้งเตือน & มีระบบแจ้งเตือนครบถ้วน & ไม่มีระบบเตือน \\ 
\hline
สแกนบาร์โค้ด / QR Code / ตัวหนังสือ & สแกนได้ทั้ง "บาร์โค้ด" และ "ข้อความบนฉลากยา" & สแกนรูปยา (AI Image Recognition) & ไม่มีสแกนบาร์โค้ด & ไม่มีสแกน & มีสแกน QR Code\\ 
\hline
% แปลภาษาข้อมูลยา & แปลข้อความยาได้ (ไทย $\leftrightarrow$ อังกฤษ) & ไม่มี & ไม่มี & ไม่มี & ไม่มี \\ 
% \hline
ข้อมูลยา & TMT และ อย. & ขึ้นกับฐานข้อมูลของ AI & อ้างอิงจากฐานข้อมูล อย. & ข้อมูลทั่วไป & อ้างอิงจากฐานข้อมูล อย.\\ 
\hline
เหมาะกับผู้สูงอายุ / ใช้งานง่าย & UI เรียบง่าย พัฒนาเพื่อผู้สูงอายุ & UI ใช้งานยาก และไม่เหมาะกับบุคคลทั่วไป & UI ใช้งานง่าย แต่ไม่อัปเดต & ระบบแจ้งเตืนมีความซับซ้อนและใช้งานยาก & UI ใช้งานยาก และยังไม่มีการอัปเดตเพิ่มเติม\\ 
\hline
บันทึกประวัติยาในเครื่อง & มีระบบเก็บประวัติยาและอาการย้อนหลัง & เก็บเฉพาะภาพยา & บันทึกข้อมูลยาได้ & เก็บข้อมูลยาและเวลาเตือน & ไม่มีระบบบันทึก \\ 
\hline
เทคโนโลยี AI / OCR & ใช้ OCR อ่านฉลากยา และตรวจบาร์โค้ด & ใช้ AI แยกภาพยา & ไม่มี & ไม่มี & ไม่มี \\ 
\hline
\end{tabular}%
}
\captionsetup{justification=centering, name=ตาราง}
\caption{เปรียบเทียบแอปพลิเคชันในตลาด}
\label{table:app-comparison}
\end{table}
จากการเปรียบเทียบข้างต้น จะเห็นได้ว่า CapYaDoo มีข้อได้เปรียบในหลายด้าน ได้แก่

\subsubsection{ข้อได้เปรียบของ CapYaDoo}
\begin{enumerate}[leftmargin=4em]
    \item สามารถสแกนได้ทั้งบาร์โค้ดและข้อความ ทำให้ง่ายต่อการใช้งาน
    \item สามารถจัดการข้อมูลยาและอาการได้อย่างครบถ้วน ได้แก่ เพิ่ม แก้ไข และลบข้อมูลยาที่กำลังใช้อยู่
    \item รองรับการอ่านข้อมูลยาให้ผู้ใช้ฟัง (Text-to-Speech)
    \item สามารถตั้งแจ้งเตือนการทานยาได้ และสามารถแจ้งเตือนผู้ใช้เมื่อถึงเวลา
\end{enumerate}
\subsubsection{ข้อจำกัดของ CapYaDoo}
\begin{enumerate}[leftmargin=4em]
    \item รองรับเพียงระบบปฏิบัติการ Android อย่างเดียว
    \item เนื่องจากยังไม่มีการร่วมมือกับหน่วยงานใด ทำให้มีข้อจำกัดในการพัฒนาแอปพลิเคชันให้ครอบคลุมทุกฟีเจอร์
\end{enumerate}
% \subsubsection{จุดเด่นที่เหมือนกัน}
% \begin{enumerate}[leftmargin=2em]
%     \item \textbf{การค้นหาข้อมูลยา:} แอปพลิเคชันส่วนใหญ่เน้นการค้นหาด้วยชื่อยาเป็นหลัก
%     \item \textbf{การแจ้งเตือนการทานยา:} มีแจ้งเตือนการทานยาตามกำหนด
%     \item \textbf{การบันทึกประวัติยา:} มีระบบบันทึก/เก็บประวัติข้อมูลยาที่เคยใช้งานในเครื่อง
%     \item แอปพลิเคชันส่วนใหญ่ใช้ฐานข้อมูลยาจาก อย. เป็นหลัก ทำให้ข้อมูลมีความน่าเชื่อถือ
%     \item แอปพลิเคชันส่วนใหญ่เน้นการค้นหาด้วยชื่อยาเป็นหลัก และสามารถดูรายละเอียดข้อมูลยาได้
%     \item 
% \end{enumerate}

\subsection{สรุปการเปรียบเทียบ}
จากการเปรียบเทียบแอปพลิเคชันที่มีอยู่ในปัจจุบัน พบว่าแต่ละแอปพลิเคชันมีฟีเจอร์ที่โดดเด่นแตกต่างกันไป ส่งผลให้ผู้ใช้ต้องสลับใช้งานหลายแอปพลิเคชันจึงจะสามารถจัดการข้อมูลการทานยาได้อย่างครบถ้วน แอปพลิเคชัน CapYaDoo จึงถูกพัฒนาขึ้นเพื่อตอบโจทย์ปัญหานี้ โดยรวบรวมฟังก์ชันสำคัญทั้งหมด ไม่ว่าจะเป็นการแจ้งเตือน การติดตามประวัติการทานยา การจัดเก็บข้อมูลยา และการบันทึกอาการ มาไว้ภายในแอปเดียว ช่วยให้ผู้ใช้สามารถจัดการด้านการทานยาได้สะดวกและมีประสิทธิภาพมากยิ่งขึ้น

นอกจากนี้ แอปพลิเคชันจำนวนมากยังไม่ได้ออกแบบโดยคำนึงถึงความต้องการเฉพาะของผู้สูงอายุ เช่น การอ่านตัวอักษรยาก การใช้งานซับซ้อน หรือองค์ประกอบบนหน้าจอที่ไม่เป็นมิตรต่อผู้ใช้งาน ทำให้ผู้สูงอายุประสบปัญหาในการใช้งานจริง รวมทั้งบางแอปพลิเคชันไม่มีการอัปเดตอย่างต่อเนื่อง ทำให้ดีไซน์และประสบการณ์ใช้งานล้าสมัย ไม่สอดคล้องกับพฤติกรรมผู้ใช้ในปัจจุบัน

ด้วยเหตุนี้ CapYaDoo จึงมุ่งเน้นการออกแบบที่เหมาะสมกับผู้สูงอายุ ควบคู่กับการรวมฟังก์ชันที่จำเป็นทั้งหมดเข้าด้วยกัน เพื่อให้เป็นแอปพลิเคชันที่ตอบโจทย์การจัดการการทานยาอย่างครบวงจร
% \subsection{การวิเคราะห์ช่องว่างในตลาด}

% จากตารางเปรียบเทียบจะเห็นว่า แอปพลิเคชันในตลาดส่วนใหญ่เน้นฟังก์ชันพื้นฐานอย่างการแจ้งเตือนและการสแกน Barcode แต่ยังขาดฟังก์ชันสำคัญ เช่น

% \begin{enumerate}[leftmargin=2em]
%     \item \textbf{การสแกนตัวอักษร (OCR)} สำหรับยาที่ไม่มี Barcode
%     \item \textbf{การเพิ่มยาที่กำลังทาน} และติดตามประวัติการใช้ยา
%     \item \textbf{การติดตามอาการ} หลังจากการทานยา
%     \item \textbf{การแปลภาษา} และ Text-to-Speech สำหรับผู้สูงอายุ
%     \item \textbf{การ Export รายงาน} ประวัติยาและอาการ
% \end{itemize}

% เนื่องจากฟังก์ชันเหล่านี้ไม่ได้รวมอยู่ในแอปพลิเคชันเดียว แอปของเราจึงรวบรวมฟังก์ชันพื้นฐานและเพิ่มเติมฟังก์ชันที่จำเป็นเพื่อความครบถ้วนในการใช้งาน โดยเฉพาะการตอบสนองความต้องการของผู้สูงอายุที่มีข้อจำกัดทางด้านการมองเห็นและการได้ยิน

% \section{สรุปและการนำไปประยุกต์ใช้}
% \label{sec:related-work-summary}

% จากการศึกษางานวิจัยและโครงการที่เกี่ยวข้องทั้ง 3 งาน รวมถึงการเปรียบเทียบแอปพลิเคชันที่มีอยู่ในตลาด สามารถสรุปประเด็นสำคัญที่จะนำมาประยุกต์ใช้ในโครงงานนี้ได้ดังนี้

% \subsection{แนวทางการพัฒนาที่ได้จากงานวิจัย}
% \begin{enumerate}[leftmargin=2em]
%     \item \textbf{การออกแบบที่เหมาะกับผู้สูงอายุ} จากงานวิจัยที่ 1 แสดงให้เห็นถึงความสำคัญของการออกแบบ UI/UX ที่เรียบง่าย ใช้ตัวอักษรขนาดใหญ่ และมีความชัดเจน
    
%     \item \textbf{ระบบแจ้งเตือนและติดตามผล} จากงานวิจัยที่ 2 แสดงให้เห็นว่าการมีระบบแจ้งเตือนและติดตามผลอย่างเป็นระบบสามารถเพิ่มความสม่ำเสมอในการทานยาได้อย่างมีนัยสำคัญ
    
%     \item \textbf{เทคโนโลยี OCR} จากงานวิจัยที่ 3 แสดงให้เห็นถึงประสิทธิภาพของ OCR ในการอ่านข้อมูลยา แต่ต้องคำนึงถึงความยืดหยุ่นในการรองรับรูปแบบฉลากที่แตกต่างกัน
% \end{enumerate}

% \subsection{ข้อจำกัดที่ต้องหลีกเลี่ยง}
% \begin{enumerate}[leftmargin=2em]
%     \item หลีกเลี่ยงการพึ่งพาแพลตฟอร์มเฉพาะที่อาจมีการเปลี่ยนแปลงหรือยุติบริการ
%     \item ต้องพัฒนาระบบ OCR ให้มีความแม่นยำและรองรับรูปแบบฉลากที่หลากหลาย
%     \item ต้องแก้ไขปัญหา delay ในการประมวลผลให้มีประสิทธิภาพมากขึ้น
% \end{enumerate}

% \subsection{คุณสมบัติที่จะนำมาพัฒนาในโครงงาน}
% โครงงานนี้จะนำแนวคิดและข้อดีจากงานวิจัยที่เกี่ยวข้องมาประยุกต์ใช้ โดยมีจุดเด่นดังนี้
% \begin{enumerate}[leftmargin=2em]
%     \item ใช้เทคโนโลยี OCR ในการอ่านฉลากยาพร้อมกับฐานข้อมูลยาจาก Thai Medicine Terminology (TMT) ที่มีความน่าเชื่อถือ
%     \item พัฒนาระบบแจ้งเตือนที่หลากหลาย เช่น เสียง, การสั่น เพื่อรองรับผู้สูงอายุที่มีข้อจำกัดทางด้านการได้ยินและการมองเห็น
%     \item สร้างระบบบันทึกและติดตามประวัติการทานยาที่ครบถ้วน
%     \item ออกแบบ UI/UX ที่เหมาะสมกับผู้สูงอายุโดยเฉพาะ
% \end{enumerate} 

\section{ประสบการณ์การพัฒนาที่ผ่านมา}
\label{sec:development-experience}

เพื่อให้เห็นถึงประสบการณ์และความสามารถของทีมผู้พัฒนาในการสร้างแอปพลิเคชัน จึงได้นำเสนอโครงการที่ผ่านมาที่เกี่ยวข้องกับการพัฒนาแอปพลิเคชันด้วย Flutter Framework โดยมีรายละเอียดดังนี้

\subsection{Project 1: Application Chat Music Player}
\label{sec:chat-music-player}

\textbf{วัตถุประสงค์:}
\begin{enumerate}[leftmargin=4em]
    \item เพื่อให้ผู้ใช้งานเกิดความผ่อนคลายและสร้างความสุขระหว่างการใช้งานให้กับผู้ใช้ได้
    \item เพื่อให้ผู้ใช้งานได้พูดคุยและแลกเปลี่ยนความคิดเห็นกับผู้ใช้คนอื่น ๆ ได้
\end{enumerate}

\textbf{บทบาทและสิ่งที่ทำ:}
\begin{enumerate}[leftmargin=4em]
    \item \textbf{นาย ฤทธิชล พลราช:} มีหน้าที่ในการออกแบบ UX/UI และทำการพัฒนาแอปพลิเคชัน
    \item \textbf{นางสาว จิรัชญา ราชพลแสน:} มีหน้าที่ในการจัดทำรายงานและพัฒนาแอปพลิเคชัน
\end{enumerate}

\textbf{เทคโนโลยีที่ใช้:}
\begin{enumerate}[leftmargin=4em]
    \item ใช้ภาษา Dart และ Flutter framework
    \item Backend ใช้ Firebase Firestore, Firebase Authentication
    \item Package ที่ใช้คือ audioplayer
\end{enumerate}

\subsection{Project 2: Application Location Discovery}
\label{sec:location-discovery}

\textbf{วัตถุประสงค์หลัก:}
\begin{enumerate}[leftmargin=4em]
    \item เพื่อให้ผู้ใช้งานเลือก/ไม่เลือก สถานที่พักผ่อน อ่านหนังสือภายในมหาวิทยาลัย
    \item เพื่อให้ผู้ใช้งานใช้ข้อมูลในการตัดสินเลือก เช่น สถานที่ที่อยู่ใกล้เรา สิ่งอำนวยความสะดวกในสถานที่นั้น
\end{enumerate}

\textbf{บทบาทและสิ่งที่ทำ:}
\begin{enumerate}[leftmargin=4em]
    \item \textbf{นางสาว ปรารถนา สุภาวงค์:} ทำหน้าที่ในการออกแบบแอปพลิเคชัน Logo, UI และพัฒนาแอปพลิเคชัน
\end{enumerate}

\textbf{เทคโนโลยีที่ใช้:}
\begin{enumerate}[leftmargin=4em]
    \item ใช้ภาษา Dart และ Flutter framework
    \item Backend ใช้ Firebase Firestore, Firebase Authentication
\end{enumerate}

\subsection{บทเรียนที่ได้จากประสบการณ์การพัฒนา}

จากการพัฒนาแอปพลิเคชันทั้งสองโครงการ ทีมผู้พัฒนาได้เรียนรู้และสะสมประสบการณ์ในด้านต่างๆ ดังนี้

\subsubsection{ด้านการออกแบบ UI/UX}
\begin{enumerate}[leftmargin=4em]
    \item การออกแบบส่วนติดต่อผู้ใช้ที่ใช้งานง่ายและสวยงาม
    \item การเลือกใช้สีและฟอนต์ที่เหมาะสมกับกลุ่มผู้ใช้
    \item การจัดวางองค์ประกอบบนหน้าจอให้เป็นระเบียบและเข้าถึงได้ง่าย
\end{enumerate}

\subsubsection{ด้านการพัฒนาแอปพลิเคชัน}
\begin{enumerate}[leftmargin=4em]
    \item การใช้ Flutter Framework ในการพัฒนาแอปพลิเคชันข้ามแพลตฟอร์ม
    \item การจัดการ State Management ในแอปพลิเคชัน
    \item การใช้งาน Package ต่างๆ เพื่อเพิ่มฟังก์ชันให้กับแอปพลิเคชัน
\end{enumerate}

% \subsubsection{ด้านการจัดการฐานข้อมูล}
% \begin{enumerate}[leftmargin=4em]
%     \item การใช้ Firebase Firestore ในการจัดเก็บข้อมูล
%     \item การจัดการ Authentication และ User Management
%     \item การออกแบบโครงสร้างข้อมูลให้เหมาะสมกับการใช้งาน
% \end{enumerate}

\subsubsection{ด้านการทำงานเป็นทีม}
\begin{enumerate}[leftmargin=4em]
    \item การแบ่งหน้าที่และความรับผิดชอบในทีม
    \item การสื่อสารและประสานงานระหว่างสมาชิกในทีม
    \item การจัดการเวลาและกำหนดการในการพัฒนา
\end{enumerate}

ประสบการณ์เหล่านี้จะเป็นประโยชน์อย่างมากในการพัฒนาโครงงาน CapYaDoo เนื่องจากทีมผู้พัฒนาได้มีความคุ้นเคยกับ Flutter Framework และการทำงานร่วมกันมาแล้ว
