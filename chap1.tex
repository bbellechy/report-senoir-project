\chapter{บทนำ}
\label{Ch:Introduction}

\section{ที่มาและความสำคัญของปัญหา}
\label{Sec:ProblemStatement}

ประเทศไทยกำลังเผชิญหน้ากับการเปลี่ยนแปลงทางประชากรครั้งใหญ่ โดยกำลังก้าวเข้าสู่ "สังคมสูงวัยระดับสุดยอด" (Super-Aged Society) ซึ่งเป็นผลมาจากสัดส่วนประชากรผู้สูงอายุที่เพิ่มขึ้นอย่างรวดเร็ว ข้อมูลระบุว่าในปี พ.ศ. 2548 ไทยได้เข้าสู่ "สังคมสูงวัย" (Aged Society) ที่มีประชากรอายุเกิน 60 ปี มากกว่าร้อยละ 10 ของประชากรทั้งหมด และในปี พ.ศ. 2566 ได้ก้าวเข้าสู่ "สังคมสูงวัยอย่างสมบูรณ์" (Complete Aged Society) ด้วยสัดส่วนประชากรกลุ่มนี้ที่เพิ่มขึ้นมากกว่าร้อยละ 20 และมีแนวโน้มที่จะเข้าสู่ "สังคมสูงวัยระดับสุดยอด" (Super-Aged Society) ในปี พ.ศ. 2576 ซึ่งจะมีประชากรสูงอายุเกิน 60 ปี มากกว่าร้อยละ 28 หรือมีผู้สูงอายุเกิน 65 ปี มากกว่าร้อยละ 20 ของประชากรทั้งหมด (กรมอนามัย, 2566)

การเข้าสู่สังคมสูงอายุอย่างรวดเร็วของประเทศไทยส่งผลให้เกิดความท้าทายด้านสุขภาพอย่างหลีกเลี่ยงไม่ได้ โดยเฉพาะอย่างยิ่งในเรื่องของ กลุ่มโรคไม่ติดต่อเรื้อรัง (Non-Communicable Diseases หรือ NCDs) ซึ่งเป็นสาเหตุการเสียชีวิตอันดับต้น ๆ ของประชากรทั่วโลก ปัจจัยเสี่ยงสำคัญ ได้แก่ การสูบบุหรี่ การไม่ออกกำลังกาย การรับประทานอาหารไม่เหมาะสม และการดื่มแอลกอฮอล์มากเกินไป ซึ่งนำไปสู่โรคหัวใจ มะเร็ง โรคระบบหายใจเรื้อรัง เบาหวาน และความดันโลหิตสูง (“NCD โรคไม่ติดต่อ” โดย ผศ.ดร.พญ.มยุรี หอมสนิท, 2557)

ผู้สูงอายุที่ป่วยด้วยโรคไม่ติดต่อเรื้อรัง (NCDs) มักต้องจัดการการใช้ยาที่ซับซ้อน เช่น การจดจำชนิดของยา เวลารับประทาน และผลข้างเคียง การลืมทานยาหรือทานผิดเวลาพบได้บ่อย และส่งผลต่อประสิทธิภาพการรักษา ขณะเดียวกัน แอปพลิเคชันสุขภาพทั่วไปยังไม่เหมาะสมกับการใช้งานของผู้สูงอายุ โดยเฉพาะในด้านการออกแบบหน้าจอ (UI) ที่ซับซ้อน

จากความท้าทายเหล่านี้ การพัฒนาแอปพลิเคชันที่สามารถแก้ไขปัญหาดังกล่าวได้จึงมีความสำคัญอย่างยิ่ง แอปพลิเคชันนี้จะทำหน้าที่เป็นเครื่องมือที่ใช้งานง่าย มีข้อมูลที่น่าเชื่อถือ และมีฟังก์ชันที่ช่วยลดความผิดพลาดในการใช้ยา เพื่อช่วยให้ทั้งผู้ป่วยสามารถจัดการสุขภาพได้อย่างมีประสิทธิภาพและปลอดภัยสูงสุดในยุคที่ประเทศไทยกำลังก้าวเข้าสู่สังคมสูงอายุอย่างเต็มตัว


\section{วัตถุประสงค์}
\label{Sec:Objective}

\begin{itemize}
    \item เพื่อพัฒนาแอปพลิเคชันสำหรับแจ้งเตือน บันทึการทานยา และบันทึกอาการหลังทานยา เพื่อช่วยให้ผู้ใช้งานจดจำและทานยาได้ถูกต้องสม่ำเสมอ พร้อมระบบเพิ่ม–แก้ไขข้อมูลยาได้ด้วยตนเอง

    \item เพื่อเพิ่มฟังก์ชันที่ตอบสนองความต้องการของผู้สูงอายุ ได้แก่
    \begin{itemize}
        \item การแจ้งเตือนเพื่อช่วยเหลือผู้สูงอายุที่มีข้อจำกัดในการได้ยินและการมองเห็นจากการใช้เสียงแจ้งเตือนที่ดังและชัดเจน และยังมีการแจ้งเตือนด้วยการสั่น
        \item การแปลงข้อความให้เป็นเสียงเพื่ออำนวยความสะดวกในการเข้าถึงข้อมูลสำหรับผู้ที่มีปัญหาด้านการมองเห็น
    \end{itemize}
    \item เพื่อสร้างฐานข้อมูลยาที่น่าเชื่อถือ โดยจะรวบรวมข้อมูลยาที่สำคัญจากจากระบบ Thai Medicine Terminology (TMT)
    \item เพื่อให้ผู้ใช้งานสามารถเพิ่ม, แก้ไข, และลบข้อมูลยาที่กำลังทานอยู่หรือบันทึกประวัติการทานยาได้
    \item เพื่อช่วยให้ผู้ใช้งานสามารถบันทึกและติดตามประวัติการทานยาในแต่ละวัน เพื่อให้ผู้ใช้งานนำข้อมูลไปปรึกษาแพทย์ได้อย่างถูกต้อง และเพื่อให้ผู้ใช้งานสามารถตรวจสอบความสม่ำเสมอในการทานยาได้อย่างแม่นยำ
\end{itemize}

\section{ขอบเขตงาน}
\label{Sec:ScopeOfWork}

\subsection{ด้านฟังก์ชันการทำงาน}
\begin{itemize}[leftmargin=2em]

    \item ผู้ใช้งานสามารถเพิ่ม, แก้ไข, และลบข้อมูลยาที่กำลังทานอยู่หรือบันทึกประวัติการทานยาได้
    \item ผู้ใช้งานสามารถค้นหาข้อมูลยาเพื่อดูรายละเอียดเพิ่มเติมได้ เช่น ชื่อยา, วิธีใช้, ผลข้างเคียง
    \item พัฒนาระบบแจ้งเตือนที่สามารถแจ้งเตือนผู้ใช้งานได้ตามเวลาที่กำหนดไว้
    \item มีระบบแปลงข้อความเป็นเสียงสำหรับการอ่านข้อมูลยาให้ผู้ใช้ฟัง
\end{itemize}

\subsection{ด้านข้อมูล}
\begin{itemize}[leftmargin=2em]

    \item รวบรวมข้อมูลยาจากระบบ Thai Medicine Terminology (TMT)
    \item จัดเก็บข้อมูลการทานยาและประวัติสุขภาพของผู้ใช้งานอย่างปลอดภัย
    \item จัดทำฐานข้อมูลที่มีความสอดคล้องกันของข้อมูลและมีความปลอดภัย
\end{itemize}

\subsection{ด้านเทคนิค}
\begin{itemize}[leftmargin=2em]

    \item พัฒนาด้วยภาษา Dart และเฟรมเวิร์ก Flutter 
    \item ใช้ฐานข้อมูล PostgreSQL สำหรับจัดเก็บข้อมูลยา และข้อมูลผู้ใช้
    \item ใช้ PySpark ในการจัดการข้อมูลจากระบบ Thai Medicine Terminology (TMT) ก่อนเก็บลงฐานข้อมูล
    \item ใช้ Kind เพื่อสร้าง Kubernetes cluster จำลองบนเครื่องพัฒนา (Local Environment) สำหรับการทดสอบการทำงานของระบบ Spring Boot Application, PySpark และ ฐานข้อมูล PostgreSQL
\end{itemize}

\subsection{เครื่องมือเสริม}
\begin{itemize}[leftmargin=2em]
    \item google\_mlkit\_text\_recognition สำหรับอ่านข้อความบนฉลากยา
    \item mobile\_scanner สำหรับสแกนบาร์โค้ด
    \item flutter\_tts สำหรับทำ Text-to-Speech
    \item flutter\_local\_notifications สำหรับแจ้งเตือน
\end{itemize}

\subsection{ด้านการศึกษา}
\begin{itemize}[leftmargin=2em]

    \item การศึกษาเกี่ยวกับการพัฒนาแอปพลิเคชันสำหรับผู้สูงอายุ
    \item ศึกษาการทำงานของ Flutter และ Dart
    \item ศึกษาการทำงานของ Spring Boot
    \item ศึกษาการทำงานของ PySpark
    \item ศึกษาการทำงานของ kind
    \item ศึกษาการทำงานของ package flutter\_tesseract\_ocr
    \item ศึกษาการทำงานของ package mobile\_scanner
    \item ศึกษาการทำงานของ package flutter\_tts
    \item ศึกษาการทำงานของ package flutter\_local\_notifications
\end{itemize}

\section{ขั้นตอนการดำเนินงาน}
\label{Sec:Procedure}
 
โครงงานนี้มีการดำเนินงานและการศึกษาที่เกี่ยวข้องแบ่งออกเป็น 6 ขั้นตอนหลัก เพื่อให้การพัฒนาเป็นไปอย่างเป็นระบบและสามารถตอบโจทย์ความต้องการของผู้ใช้งานได้จริง โดยมีรายละเอียดดังนี้

\subsection{ขั้นตอนที่ 1: การรวบรวมปัญหาและวิเคราะห์ความต้องการของผู้ใช้งาน}
ในขั้นตอนแรกของโครงงาน ได้ทำการศึกษาและวิเคราะห์ปัญหาของกลุ่มเป้าหมาย ซึ่งเป็นผู้สูงอายุที่ต้องรับประทานยาหลายชนิดในแต่ละวัน รวมถึงศึกษาข้อจำกัดทางด้านการมองเห็นและการได้ยิน จากนั้นจึงทำการเก็บรวบรวมความต้องการของผู้ใช้งาน (User Requirement) โดยการสัมภาษณ์และสำรวจความคิดเห็นจากกลุ่มผู้สูงอายุและผู้ดูแล เพื่อนำมาวิเคราะห์และกำหนดฟังก์ชันการทำงานของแอปพลิเคชันที่ตอบโจทย์ความต้องการได้อย่างแท้จริง

\subsection{ขั้นตอนที่ 2: การออกแบบและพัฒนาระบบในส่วนของหน้าบ้าน (Frontend)}
ในขั้นตอนนี้ได้ทำการออกแบบและพัฒนาแอปพลิเคชันบนระบบปฏิบัติการ Android โดยใช้ Flutter Framework ร่วมกับภาษา Dart เพื่อสร้างส่วนติดต่อผู้ใช้งาน (User Interface) ที่เหมาะสมกับผู้สูงอายุ โดยเน้นความเรียบง่าย ใช้งานง่าย มีตัวอักษรขนาดใหญ่ ปุ่มที่มีขนาดใหญ่และเห็นได้ชัดเจน รวมถึงการเลือกใช้สีที่มีความคมชัดเพื่อให้ผู้สูงอายุมองเห็นได้ง่าย นอกจากนี้ยังได้พัฒนาฟีเจอร์การแจ้งเตือนด้วยเสียง การสั่น และการแปลงข้อความเป็นเสียง (Text-to-Speech) เพื่ออำนวยความสะดวกให้กับผู้ใช้งาน

\subsection{ขั้นตอนที่ 3: การพัฒนาระบบในส่วนของหลังบ้าน (Backend)}
ในส่วนของหลังบ้าน ใช้ Spring Boot Framework ในการสร้าง REST API ที่เชื่อมต่อระหว่างแอปพลิเคชันกับฐานข้อมูล PostgreSQL ซึ่งทำหน้าที่เก็บข้อมูลยา ข้อมูลผู้ใช้ ตารางการทานยา และข้อมูลการบันทึกประวัติการรับประทานยา โดยออกแบบ API endpoints ให้รองรับการทำงานต่างๆ เช่น การเพิ่ม แก้ไข และลบข้อมูลยา การตั้งค่าการแจ้งเตือน และการดึงข้อมูลยาจากฐานข้อมูล นอกจากนี้ เพื่อให้ระบบสามารถทดสอบได้ในสภาพแวดล้อมเสมือนจริง ได้มีการใช้เครื่องมือ Kind (Kubernetes in Docker) ในการจำลองคลัสเตอร์ Kubernetes ภายในเครื่อง เพื่อทดสอบการทำงานของระบบในรูปแบบ Container

\subsection{ขั้นตอนที่ 4: การศึกษาและพัฒนา Data Pipeline ด้วย PySpark และ Kind}
ขั้นตอนนี้เป็นการออกแบบและพัฒนากระบวนการ Data Pipeline เพื่อจัดการข้อมูลยาจากแหล่งข้อมูลภายนอก โดยเฉพาะข้อมูลจากระบบ Thai Medicine Terminology (TMT) จากสำนักพัฒนามาตรฐานระบบข้อมูลสุขภาพไทย ซึ่งให้ข้อมูลยาในรูปแบบไฟล์ Excel โดยใช้ PySpark ในการประมวลผล ทำความสะอาดข้อมูล (Data Cleaning) และแปลงข้อมูลให้อยู่ในรูปแบบที่เหมาะสมก่อนนำเข้าฐานข้อมูล PostgreSQL เพื่อให้ระบบ Data Pipeline สามารถทำงานร่วมกับส่วน Backend และ Database ได้อย่างราบรื่น ได้มีการนำ Kind มาใช้ควบคุมการทำงานของ Container ทั้งหมด โดยภายใน Cluster จะประกอบด้วย
\begin{itemize}[leftmargin=2em]
    \item Container ของ PySpark สำหรับประมวลผลและจัดการข้อมูลยา
    \item Container ของ Spring Boot สำหรับให้บริการ API
    \item Container ของ PostgreSQL สำหรับเก็บข้อมูลที่ผ่านการประมวลผลแล้ว
\end{itemize}

\subsection{ขั้นตอนที่ 5: การทดสอบและปรับปรุงระบบ}
ในขั้นตอนนี้ทำการทดสอบระบบทั้งหมดเพื่อตรวจสอบความถูกต้องและประสิทธิภาพของแอปพลิเคชัน โดยแบ่งการทดสอบออกเป็น Unit Testing สำหรับทดสอบฟังก์ชันแต่ละส่วน Integration Testing สำหรับทดสอบการทำงานร่วมกันระหว่างส่วนต่างๆ และ User Acceptance Testing (UAT) โดยให้กลุ่มผู้สูงอายุทดลองใช้งานจริง จากนั้นรวบรวม Feedback และข้อเสนอแนะจากผู้ทดสอบเพื่อนำมาปรับปรุงแก้ไขระบบให้มีความสมบูรณ์และตอบสนองความต้องการของผู้ใช้งานได้ดียิ่งขึ้น

\subsection{ขั้นตอนที่ 6: การจัดทำเอกสารและนำเสนอ}
ขั้นตอนสุดท้ายเป็นการจัดทำเอกสารรายงานโครงงาน โดยรวบรวมข้อมูลทั้งหมดตั้งแต่การศึกษาปัญหา การวิเคราะห์ความต้องการ กระบวนการพัฒนา ผลการทดสอบ และข้อสรุป พร้อมทั้งเตรียมเอกสารคู่มือการใช้งานสำหรับผู้ใช้ (User Manual) และเอกสารทางเทคนิคสำหรับผู้พัฒนา (Technical Documentation) นอกจากนี้ยังได้เตรียมการนำเสนอผลงานโครงงานต่ออาจารย์ที่ปรึกษาและคณะกรรมการ พร้อมทั้งสาธิตการทำงานของแอปพลิเคชันให้เห็นถึงฟังก์ชันและความสามารถของระบบอย่างชัดเจน

\begin{figure}[H]
    \centering
    \includegraphics[width=0.9\textwidth]{project_plan.png}
    \caption{แผนการดำเนินโครงงาน}
    \label{fig:project_plan}
\end{figure}

\section{ประโยชน์ที่คาดว่าจะได้รับ}
\label{Sec:ExpectedBenefit}
\begin{itemize}
    \item ผู้ใช้งานจะสามารถทานยาได้อย่างถูกต้องและสม่ำเสมอตามคำสั่งแพทย์ 
    \item ผู้สูงอายุและผู้ที่มีปัญหาด้านการมองเห็นและการได้ยินจะสามารถเข้าถึงข้อมูลยาได้อย่างปลอดภัยและง่ายมากยิ่งขึ้น
    \item ผู้ใช้งานจะได้รับข้อมูลยาที่ถูกต้องและมีแหล่งอ้างอิงที่ชัดเจน
    \item ช่วยให้ผู้ใช้งานลดความกังวลในการจัดการยาด้วยตัวเองและไม่ลืมที่จะทานยา
    \item ช่วยให้การให้ข้อมูลการทานยาแก่แพทย์เป็นไปอย่างแม่นยำและถูกต้อง ส่งผลให้การวินิจฉัยและการรักษามีประสิทธิภาพยิ่งขึ้น
\end{itemize}

