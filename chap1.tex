\chapter{บทนำ}
\label{Ch:Introduction}

\section{ที่มาและความสำคัญของปัญหา}
\label{Sec:ProblemStatement}

ประเทศไทยกำลังเผชิญหน้ากับการเปลี่ยนแปลงทางประชากรครั้งใหญ่ โดยกำลังก้าวเข้าสู่ "สังคมสูงวัยระดับสุดยอด" (Super-Aged Society) ซึ่งเป็นผลมาจากสัดส่วนประชากรผู้สูงอายุที่เพิ่มขึ้นอย่างรวดเร็ว ข้อมูลระบุว่าในปี พ.ศ. 2548 ไทยได้เข้าสู่ "สังคมสูงวัย" (Aged Society) ที่มีประชากรอายุเกิน 60 ปี มากกว่าร้อยละ 10 ของประชากรทั้งหมด และในปี พ.ศ. 2566 ได้ก้าวเข้าสู่ "สังคมสูงวัยอย่างสมบูรณ์" (Complete Aged Society) ด้วยสัดส่วนประชากรกลุ่มนี้ที่เพิ่มขึ้นมากกว่าร้อยละ 20 และมีแนวโน้มที่จะเข้าสู่ "สังคมสูงวัยระดับสุดยอด" (Super-Aged Society) ในปี พ.ศ. 2576 ซึ่งจะมีประชากรสูงอายุเกิน 60 ปี มากกว่าร้อยละ 28 หรือมีผู้สูงอายุเกิน 65 ปี มากกว่าร้อยละ 20 ของประชากรทั้งหมด \cite{anamai2025}

การเข้าสู่สังคมสูงอายุอย่างรวดเร็วของประเทศไทยส่งผลให้เกิดความท้าทายด้านสุขภาพอย่างหลีกเลี่ยงไม่ได้ โดยเฉพาะอย่างยิ่งในเรื่องของ กลุ่มโรคไม่ติดต่อเรื้อรัง (Non-Communicable Diseases หรือ NCDs) ซึ่งเป็นสาเหตุการเสียชีวิตอันดับต้น ๆ ของประชากรทั่วโลก ปัจจัยเสี่ยงสำคัญ ได้แก่ การสูบบุหรี่ การไม่ออกกำลังกาย การรับประทานอาหารไม่เหมาะสม และการดื่มแอลกอฮอล์มากเกินไป ซึ่งนำไปสู่โรคหัวใจ มะเร็ง โรคระบบหายใจเรื้อรัง เบาหวาน และความดันโลหิตสูง \cite{ncd2014}

ผู้สูงอายุที่ป่วยด้วยโรคไม่ติดต่อเรื้อรัง (NCDs) มักต้องจัดการการใช้ยาที่ซับซ้อน เช่น การจดจำชนิดของยา เวลารับประทาน และผลข้างเคียง การลืมทานยาหรือทานผิดเวลาพบได้บ่อย และส่งผลต่อประสิทธิภาพการรักษา ขณะเดียวกัน แอปพลิเคชันสุขภาพทั่วไปยังไม่เหมาะสมกับการใช้งานของผู้สูงอายุ ในเรื่องของแอปพลิเคชันที่มีฟังก์ชันที่เกี่ยวข้องกับการดูข้อมูล บันทึกอาการ และการแจ้งเตือนการรับประทานยามักถูกแยกออกจากกันในคนละแอปพลิเคชัน ทำให้ผู้ใช้ต้องสลับการใช้งานหลายระบบ ส่งผลให้เกิดความยุ่งยากและเพิ่มโอกาสในการลืมหรือใช้งานผิดพลาด

จากความท้าทายเหล่านี้ การพัฒนาแอปพลิเคชันที่สามารถแก้ไขปัญหาดังกล่าวได้จึงมีความสำคัญอย่างยิ่ง แอปพลิเคชันนี้จะทำหน้าที่เป็นเครื่องมือที่ใช้งานง่าย มีข้อมูลที่น่าเชื่อถือ และมีฟังก์ชันที่ช่วยลดความผิดพลาดในการใช้ยา เพื่อช่วยให้ผู้สูงอายุสามารถจัดการสุขภาพได้อย่างมีประสิทธิภาพและปลอดภัยสูงสุดในยุคที่ประเทศไทยกำลังก้าวเข้าสู่สังคมสูงอายุอย่างเต็มตัว


\section{วัตถุประสงค์}
\label{Sec:Objective}

\begin{itemize}
    \item เพื่อให้ผู้ใช้สามารถค้นหาข้อมูลยาได้อย่างรวดเร็วและถูกต้อง ผ่านการค้นหาด้วยข้อความ การสแกนบาร์โค้ด หรือการอ่านข้อความบนฉลากยา (OCR)
    \item เพื่อเชื่อมโยงข้อมูลยาจากฐานข้อมูลที่น่าเชื่อถือ ได้แก่ ระบบ Thai Medicine Terminology (TMT) และข้อมูลจาก สำนักงานคณะกรรมการอาหารและยา (อย.) เพื่อให้ผู้ใช้งานได้รับข้อมูลยาที่ถูกต้องและครบถ้วน
    \item เพื่อให้ผู้ใช้สามารถติดตามอาการหลังการใช้ยาได้ ผ่านฟังก์ชันการบันทึกอาการหลังการใช้ยา
    \item เพื่อส่งเสริมความเข้าใจด้านข้อมูลยาแก่ผู้ใช้ จากระบบเสียงอ่านข้อมูลยา (Text-to-Speech)
    \item เพื่อพัฒนาแอปพลิเคชันที่ช่วยให้ผู้ใช้สามารถจัดการข้อมูลยาที่รับประทานได้อย่างมีประสิทธิภาพ
\end{itemize}

\section{ขอบเขตงาน}
\label{Sec:ScopeOfWork}

\subsection{ด้านฟังก์ชันการทำงาน}
\begin{itemize}[leftmargin=2em]

    \item ผู้ใช้งานสามารถเพิ่ม, แก้ไข และลบข้อมูลยาที่ทานได้
    \item ผู้ใช้งานสามารถเพิ่ม, แก้ไข และลบข้อมูลอาการหลังทานยาได้
    \item ผู้ใช้งานสามารถค้นหาข้อมูลยาผ่านการสแกนบาร์โค้ด, สแกนฉลากยา และกรอกข้อมูลยา เพื่อดูรายละเอียดเพิ่มเติมได้
    \item ผู้ใช้งานสามารถใช้ฟังก์ชันการแปลงข้อความเป็นเสียงเพื่อเข้าถึงข้อมูลยาที่ผ่านการค้นหาได้
    \item ผู้ใช้งานสามารถตั้งแจ้งเตือนการทานยาตามเวลาที่ต้องการได้
    \item ผู้ใช้งานสามารถเพิ่มรูปหลังทานเพื่อยืนยันการทานยาได้
\end{itemize}

\subsection{ด้านข้อมูล}
\begin{itemize}[leftmargin=2em]

    \item รวบรวมข้อมูลยาจากระบบ Thai Medicine Terminology (TMT) และสำนักงานคณะกรรมการอาหารและยา
    \item จัดเก็บข้อมูลการทานยาของผู้ใช้งานอย่างปลอดภัย
    \item จัดทำฐานข้อมูลที่มีความสอดคล้องกันของข้อมูลและมีความปลอดภัย
\end{itemize}

\subsection{ด้านเทคนิค}
\begin{itemize}[leftmargin=2em]

    \item พัฒนาด้วยภาษา Dart และเฟรมเวิร์ก Flutter 
    \item พัฒนาด้วยภาษา Java และเฟรมเวิร์ก Spring Boot
    \item ใช้ฐานข้อมูล PostgreSQL สำหรับจัดเก็บข้อมูล
    \item ใช้ PySpark ในการจัดการข้อมูลจากระบบ Thai Medicine Terminology (TMT) และจากเว็บไซต์ของทางสำนักงานคณะกรรมการอาหารและยาก่อนเก็บลงฐานข้อมูล
    \item ใช้ Kind เป็นเครื่องมือจำลอง Kubernetes บนเครื่องพัฒนา (Local Environment) เพื่อให้สามารถ deploy และจัดการหลาย container ได้ โดยมี container ดังนี้ Spring Boot Application, PySpark และ ฐานข้อมูล PostgreSQL
\end{itemize}

\subsection{เครื่องมือเสริม}
\begin{itemize}[leftmargin=2em]
    \item google\_mlkit\_text\_recognition สำหรับอ่านข้อความบนฉลากยา
    \item mobile\_scanner สำหรับสแกนบาร์โค้ด
    \item flutter\_tts สำหรับทำ Text-to-Speech
    \item flutter\_local\_notifications สำหรับแจ้งเตือน
\end{itemize}


% \cite{clickup}
\section{ประโยชน์ที่คาดว่าจะได้รับ}
\label{Sec:ExpectedBenefit}
\begin{enumerate}
    \item ทำให้เป็นแอปพลิเคชันที่มีฟังก์ชันที่จำเป็นต่อการจัดการปัญหาด้านการทานยาอย่างครบถ้วน โดยรวมฟังก์ชันการค้นหายา แจ้งเตือนการทานยา และบันทึกอาการหลังการใช้ยามาไว้ในแอปพลิเคชันเดียว ทำให้ผู้ใช้ไม่ต้องใช้งานหลายแอปพลิเคชันเพื่อจัดการปัญหาด้านการทานยา
    \item ผู้ใช้สามารถค้นหาข้อมูลยาได้อย่างรวดเร็วผ่านการค้นหาด้วยข้อความ การสแกนบาร์โค้ด หรือการอ่านข้อความบนฉลากยา (OCR)
    \item ผู้ใช้สามารถรับฟังข้อมูลยาในรูปแบบเสียงได้ ช่วยให้ผู้ใช้สามารถเข้าใจข้อมูลยาได้อย่างง่ายขึ้น
    \item ระบบแจ้งเตือนการทานยาตามเวลาที่กำหนดอย่างแม่นยำ ช่วยลดปัญหาการลืมทานยา การทานยาซ้ำ หรือทานยาไม่ตรงเวลา
    \item แอปพลิเคชันสามารถเก็บประวัติการใช้ยาและการแจ้งเตือนต่าง ๆ ในเครื่อง ทำให้ผู้ใช้สามารถเรียกดูย้อนหลังได้
\end{enumerate}

