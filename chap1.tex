\chapter{บทนำ}
\label{Ch:Introduction}

\section{ที่มาและความสำคัญของปัญหา}
\label{Sec:ProblemStatement}

ประเทศไทยกำลังเผชิญหน้ากับการเปลี่ยนแปลงทางประชากรครั้งใหญ่ โดยกำลังก้าวเข้าสู่ "สังคมสูงวัยระดับสุดยอด" (Super-Aged Society) ซึ่งเป็นผลมาจากสัดส่วนประชากรผู้สูงอายุที่เพิ่มขึ้นอย่างรวดเร็ว ข้อมูลระบุว่าในปี พ.ศ. 2548 ไทยได้เข้าสู่ "สังคมสูงวัย" (Aged Society) ที่มีประชากรอายุเกิน 60 ปี มากกว่าร้อยละ 10 ของประชากรทั้งหมด และในปี พ.ศ. 2566 ได้ก้าวเข้าสู่ "สังคมสูงวัยอย่างสมบูรณ์" (Complete Aged Society) ด้วยสัดส่วนประชากรกลุ่มนี้ที่เพิ่มขึ้นมากกว่าร้อยละ 20 และมีแนวโน้มที่จะเข้าสู่ "สังคมสูงวัยระดับสุดยอด" (Super-Aged Society) ในปี พ.ศ. 2576 ซึ่งจะมีประชากรสูงอายุเกิน 60 ปี มากกว่าร้อยละ 28 หรือมีผู้สูงอายุเกิน 65 ปี มากกว่าร้อยละ 20 ของประชากรทั้งหมด \cite{anamai2025}

การเข้าสู่สังคมสูงอายุอย่างรวดเร็วของประเทศไทยส่งผลให้เกิดความท้าทายด้านสุขภาพอย่างหลีกเลี่ยงไม่ได้ โดยเฉพาะอย่างยิ่งในเรื่องของ กลุ่มโรคไม่ติดต่อเรื้อรัง (Non-Communicable Diseases หรือ NCDs) ซึ่งเป็นสาเหตุการเสียชีวิตอันดับต้น ๆ ของประชากรทั่วโลก ปัจจัยเสี่ยงสำคัญ ได้แก่ การสูบบุหรี่ การไม่ออกกำลังกาย การรับประทานอาหารไม่เหมาะสม และการดื่มแอลกอฮอล์มากเกินไป ซึ่งนำไปสู่โรคหัวใจ มะเร็ง โรคระบบหายใจเรื้อรัง เบาหวาน และความดันโลหิตสูง \cite{ncd2014}

ผู้สูงอายุที่ป่วยด้วยโรคไม่ติดต่อเรื้อรัง (NCDs) มักต้องจัดการการใช้ยาที่ซับซ้อน เช่น การจดจำชนิดของยา เวลารับประทาน และผลข้างเคียง การลืมทานยาหรือทานผิดเวลาพบได้บ่อย และส่งผลต่อประสิทธิภาพการรักษา ขณะเดียวกัน แอปพลิเคชันสุขภาพทั่วไปยังไม่เหมาะสมกับการใช้งานของผู้สูงอายุ โดยเฉพาะในด้านการออกแบบหน้าจอ (UI) ที่ซับซ้อน

จากความท้าทายเหล่านี้ การพัฒนาแอปพลิเคชันที่สามารถแก้ไขปัญหาดังกล่าวได้จึงมีความสำคัญอย่างยิ่ง แอปพลิเคชันนี้จะทำหน้าที่เป็นเครื่องมือที่ใช้งานง่าย มีข้อมูลที่น่าเชื่อถือ และมีฟังก์ชันที่ช่วยลดความผิดพลาดในการใช้ยา เพื่อช่วยให้ทั้งผู้ป่วยสามารถจัดการสุขภาพได้อย่างมีประสิทธิภาพและปลอดภัยสูงสุดในยุคที่ประเทศไทยกำลังก้าวเข้าสู่สังคมสูงอายุอย่างเต็มตัว


\section{วัตถุประสงค์}
\label{Sec:Objective}

\begin{itemize}
    \item เพื่อช่วยให้ผู้ใช้สามารถจัดการตารางการทานยาได้อย่างมีประสิทธิภาพ
    \item เพื่อให้ผู้ใช้สามารถค้นหาข้อมูลยาได้อย่างรวดเร็วและถูกต้อง ผ่านการค้นหาด้วยข้อความ การสแกนบาร์โค้ด หรือการอ่านข้อความบนฉลากยา (OCR)
    \item เพื่อสร้างฐานข้อมูลยาที่น่าเชื่อถือ โดยจะรวบรวมข้อมูลยาที่สำคัญจากจากระบบ Thai Medicine Terminology (TMT) \cite{tmt2013}
    \item เพื่อให้ผู้ใช้สามารถติดตามอาการหลังการใช้ยาได้อย่างเป็นระบบ ผ่านฟังก์ชันการบันทึกอาการและผลข้างเคียงภายหลังการใช้ยา
    \item เพื่อส่งเสริมความเข้าใจด้านข้อมูลยาแก่ผู้ใช้ จากระบบเสียงอ่านข้อมูลยา (Text-to-Speech)
    \item เพื่อช่วยให้ผู้ใช้สามารถบันทึกและจัดเก็บประวัติการใช้ยาในเครื่องของตนเองได้
\end{itemize}

\section{ขอบเขตงาน}
\label{Sec:ScopeOfWork}

\subsection{ด้านฟังก์ชันการทำงาน}
\begin{itemize}[leftmargin=2em]

    \item ผู้ใช้งานสามารถเพิ่ม, แก้ไข และลบข้อมูลยาที่ทานได้
    \item ผู้ใช้งานสามารถเพิ่ม, แก้ไข และลบข้อมูลผลข้างคียงหลังทานยาได้
    \item ผู้ใช้งานสามารถค้นหาข้อมูลยาเพื่อดูรายละเอียดเพิ่มเติมได้ เช่น ชื่อยา, วิธีใช้, ผลข้างเคียง
    \item ผู้ใช้งานสามารถตั้งแจ้งเตือนการทานยาตามเวลาที่ต้องการได้
    \item ผู้ใช้งานสามารถเพิ่มรูปหลังทานเพื่อยืนยันการทาได้
    \item พัฒนาระบบแปลงข้อความเป็นเสียงสำหรับการอ่านข้อมูลยาให้ผู้ใช้ฟัง (Text-to-Speech)
\end{itemize}

\subsection{ด้านข้อมูล}
\begin{itemize}[leftmargin=2em]

    \item รวบรวมข้อมูลยาจากระบบ Thai Medicine Terminology (TMT)
    \item จัดเก็บข้อมูลการทานยาและประวัติสุขภาพของผู้ใช้งานอย่างปลอดภัย
    \item จัดทำฐานข้อมูลที่มีความสอดคล้องกันของข้อมูลและมีความปลอดภัย
\end{itemize}

\subsection{ด้านเทคนิค}
\begin{itemize}[leftmargin=2em]

    \item พัฒนาด้วยภาษา Dart และเฟรมเวิร์ก Flutter 
    \item ใช้ฐานข้อมูล PostgreSQL สำหรับจัดเก็บข้อมูลยา และข้อมูลผู้ใช้
    \item ใช้ PySpark ในการจัดการข้อมูลจากระบบ Thai Medicine Terminology (TMT) ก่อนเก็บลงฐานข้อมูล
    \item ใช้ Kind เพื่อสร้าง Kubernetes cluster จำลองบนเครื่องพัฒนา (Local Environment) สำหรับการทดสอบการทำงานของระบบ Spring Boot Application, PySpark และ ฐานข้อมูล PostgreSQL
\end{itemize}

\subsection{เครื่องมือเสริม}
\begin{itemize}[leftmargin=2em]
    \item google\_mlkit\_text\_recognition สำหรับอ่านข้อความบนฉลากยา
    \item mobile\_scanner สำหรับสแกนบาร์โค้ด
    \item flutter\_tts สำหรับทำ Text-to-Speech
    \item flutter\_local\_notifications สำหรับแจ้งเตือน
\end{itemize}

\subsection{ด้านการศึกษา}
\begin{itemize}[leftmargin=2em]

    \item การศึกษาเกี่ยวกับการพัฒนาแอปพลิเคชันสำหรับผู้สูงอายุ
    \item ศึกษาการทำงานของ Flutter และ Dart
    \item ศึกษาการทำงานของ Spring Boot
    \item ศึกษาการทำงานของ PySpark
    \item ศึกษาการทำงานของ kind
    \item ศึกษาการทำงานของ package google\_mlkit\_text\_recognition
    \item ศึกษาการทำงานของ package mobile\_scanner
    \item ศึกษาการทำงานของ package flutter\_tts
    \item ศึกษาการทำงานของ package flutter\_local\_notifications
\end{itemize}

\section{ขั้นตอนการดำเนินงาน}
\label{Sec:Procedure}
 
โครงงานนี้มีการดำเนินงานและการศึกษาที่เกี่ยวข้องแบ่งออกเป็น 6 ขั้นตอนหลัก เพื่อให้การพัฒนาเป็นไปอย่างเป็นระบบและสามารถตอบโจทย์ความต้องการของผู้ใช้งานได้จริง โดยมีรายละเอียดดังนี้

\subsection{ขั้นตอนที่ 1: การรวบรวมปัญหา วิเคราะห์ความต้องการของผู้ใช้งานและการออกแบบ}
ในขั้นตอนแรกของโครงงาน ได้ทำการศึกษาและวิเคราะห์ปัญหาของกลุ่มเป้าหมาย ซึ่งเป็นผู้สูงอายุที่ต้องรับประทานยาหลายชนิดในแต่ละวัน รวมถึงศึกษาข้อจำกัดทางด้านการมองเห็นและการได้ยิน จากนั้นจึงทำการเก็บรวบรวมความต้องการของผู้ใช้งาน (User Requirement) โดยการศึกษาแอปพลิเคชันที่มีอยู่ในตลาด เพื่อดูความคิดเห็นจากผู้ใช้ว่ามีปัญหาหรือข้อจำกัดในส่วนใด และยังขาดฟังก์ชันใดบ้าง เพื่อนำมาวิเคราะห์และกำหนดฟังก์ชันการทำงานของแอปพลิเคชันให้ตอบโจทย์ความต้องการได้อย่างแท้จริง

เมื่อได้ความต้องการของผู้ใช้งานแล้ว ขั้นตอนต่อมาคือการออกแบบระบบ (System Design) เพื่อให้เห็นภาพรวมและโครงสร้างของระบบก่อนการพัฒนา โดยประกอบด้วยส่วนสำคัญดังนี้

\begin{enumerate}[leftmargin=2em]
    \item \textbf{การออกแบบส่วนติดต่อผู้ใช้ (UI Design) ด้วย Figma} \\
    ใช้โปรแกรม \textit{Figma} ในการออกแบบส่วนติดต่อผู้ใช้ (User Interface) เพื่อสร้างต้นแบบ (Prototype) ของหน้าจอแอปพลิเคชัน โดยเน้นการออกแบบให้ใช้งานง่าย เหมาะกับผู้สูงอายุ และมีสีสันที่มองเห็นได้ชัดเจน พร้อมจัดวางองค์ประกอบให้อยู่ในตำแหน่งที่เข้าถึงได้สะดวก

    \item \textbf{การออกแบบฐานข้อมูล (Database Design) ด้วย ERD (Entity Relationship Diagram)} \\
    ทำการออกแบบแผนภาพความสัมพันธ์ของข้อมูล (ERD) เพื่อแสดงความเชื่อมโยงระหว่างข้อมูลยา ผู้ใช้งาน และประวัติการทานยา ซึ่งช่วยให้การจัดเก็บและเรียกใช้งานข้อมูลภายในระบบมีความเป็นระเบียบและสัมพันธ์กันอย่างถูกต้อง

    \item \textbf{การแยกฟังก์ชันการทำงานของระบบ (Functional Decomposition)} \\
    ทำการแตกย่อยฟังก์ชันหลักของระบบออกเป็นส่วนย่อย เพื่อช่วยให้เห็นภาพรวมของระบบอย่างเป็นลำดับชั้น และง่ายต่อการแบ่งงานพัฒนาในแต่ละส่วน
    \item \textbf{การออกแบบกระบวนการทำงานของระบบ (Data Flow Diagram: DFD)} \\
    สร้างแผนภาพแสดงการไหลของข้อมูล (DFD) เพื่อแสดงการทำงานของระบบในแต่ละกระบวนการ โดยแสดงให้เห็นถึงการรับข้อมูลจากผู้ใช้ การประมวลผล และการจัดเก็บในฐานข้อมูลอย่างเป็นระบบ
\end{enumerate}
\subsection{ขั้นตอนที่ 2: พัฒนาระบบในส่วนของหน้าบ้าน (Frontend)}
ในขั้นตอนนี้ได้ทำการออกแบบและพัฒนาแอปพลิเคชันบนระบบปฏิบัติการ Android โดยใช้ Flutter Framework ร่วมกับภาษา Dart โดยพัฒนาหน้าจอต่าง ๆ เช่น หน้าลงทะเบียนและเข้าสู่ระบบ หน้าหลักสำหรับแสดงรายการยาและเวลาในการรับประทาน รวมถึงหน้าการแจ้งเตือนการรับประทานยา หลังจากนั้นทำการเชื่อมต่อกับระบบ API เพื่อให้สามารถดึงและบันทึกข้อมูลจากส่วน Backend ได้อย่างถูกต้อง

\subsection{ขั้นตอนที่ 3: การพัฒนาระบบในส่วนของหลังบ้าน (Backend)}

ในส่วนของหลังบ้าน ใช้ Spring Boot Framework ในการสร้าง REST API ที่เชื่อมต่อระหว่างแอปพลิเคชันกับฐานข้อมูล PostgreSQL ซึ่งทำหน้าที่เก็บข้อมูลยา ข้อมูลผู้ใช้ ตารางการทานยา และข้อมูลการบันทึกประวัติการรับประทานยา โดยออกแบบ API endpoints ให้รองรับการทำงานต่างๆ เช่น การเพิ่ม แก้ไข และลบข้อมูลยา การตั้งค่าการแจ้งเตือน และการดึงข้อมูลยาจากฐานข้อมูล นอกจากนี้ เพื่อให้ระบบสามารถทดสอบได้ในสภาพแวดล้อมเสมือนจริง ได้มีการใช้เครื่องมือ Kind (Kubernetes in Docker) ในการจำลองคลัสเตอร์ Kubernetes ภายในเครื่อง เพื่อทดสอบการทำงานของระบบในรูปแบบ Container

\subsection{ขั้นตอนที่ 4: การศึกษาและพัฒนา Data Pipeline ด้วย PySpark และ Kind}
ขั้นตอนนี้เป็นการออกแบบและพัฒนากระบวนการ Data Pipeline เพื่อจัดการข้อมูลยาจากแหล่งข้อมูลภายนอก โดยเฉพาะข้อมูลจากระบบ Thai Medicine Terminology (TMT) จากสำนักพัฒนามาตรฐานระบบข้อมูลสุขภาพไทย ซึ่งให้ข้อมูลยาในรูปแบบไฟล์ Excel โดยใช้ PySpark ในการประมวลผล ทำความสะอาดข้อมูล (Data Cleaning) และแปลงข้อมูลให้อยู่ในรูปแบบที่เหมาะสมก่อนนำเข้าฐานข้อมูล PostgreSQL เพื่อให้ระบบ Data Pipeline สามารถทำงานร่วมกับส่วน Backend และ Database ได้อย่างราบรื่น ได้มีการนำ Kind มาใช้ควบคุมการทำงานของ Container ทั้งหมด โดยภายใน Cluster จะประกอบด้วย
\begin{enumerate}[leftmargin=2em]
    \item Container ของ PySpark สำหรับประมวลผลและจัดการข้อมูลยา
    \item Container ของ Spring Boot สำหรับให้บริการ API
    \item Container ของ PostgreSQL สำหรับเก็บข้อมูลที่ผ่านการประมวลผลแล้ว
\end{enumerate}

\subsection{ขั้นตอนที่ 5: การทดสอบและปรับปรุงระบบ}
ในขั้นตอนนี้ทำการทดสอบระบบทั้งหมดเพื่อตรวจสอบความถูกต้องและประสิทธิภาพของแอปพลิเคชัน โดยแบ่งการทดสอบออกเป็น Unit Testing สำหรับทดสอบฟังก์ชันแต่ละส่วน Integration Testing สำหรับทดสอบการทำงานร่วมกันระหว่างส่วนต่างๆ และ User Acceptance Testing (UAT) โดยให้กลุ่มผู้สูงอายุทดลองใช้งานจริง จากนั้นรวบรวม Feedback และข้อเสนอแนะจากผู้ทดสอบเพื่อนำมาปรับปรุงแก้ไขระบบให้มีความสมบูรณ์และตอบสนองความต้องการของผู้ใช้งานได้ดียิ่งขึ้น

% \subsection{ขั้นตอนที่ 6: การจัดทำเอกสารและนำเสนอ}
% ขั้นตอนสุดท้ายเป็นการจัดทำเอกสารรายงานโครงงาน โดยรวบรวมข้อมูลทั้งหมดตั้งแต่การศึกษาปัญหา การวิเคราะห์ความต้องการ กระบวนการพัฒนา ผลการทดสอบ และข้อสรุป พร้อมทั้งเตรียมเอกสารคู่มือการใช้งานสำหรับผู้ใช้ (User Manual) และเอกสารทางเทคนิคสำหรับผู้พัฒนา (Technical Documentation) นอกจากนี้ยังได้เตรียมการนำเสนอผลงานโครงงานต่ออาจารย์ที่ปรึกษาและคณะกรรมการ พร้อมทั้งสาธิตการทำงานของแอปพลิเคชันให้เห็นถึงฟังก์ชันและความสามารถของระบบอย่างชัดเจน

\begin{figure}[H]
    \centering
    \includegraphics[width=0.9\textwidth]{project_plan.png}
    \caption{แผนการดำเนินโครงงาน}
    \label{fig:project_plan}
\end{figure}
% \cite{clickup}
\section{ประโยชน์ที่คาดว่าจะได้รับ}
\label{Sec:ExpectedBenefit}
\begin{enumerate}
    \item ผู้ใช้สามารถค้นหาข้อมูลยาได้อย่างรวดเร็วผ่านการค้นหาด้วยข้อความ การสแกนบาร์โค้ด หรือการอ่านข้อความบนฉลากยา (OCR)
    \item ระบบแจ้งเตือนการทานยาตามเวลาที่กำหนดอย่างแม่นยำ ช่วยลดปัญหาการลืมทานยา การทานยาซ้ำ หรือทานยาไม่ตรงเวลา
    \item ผู้ใช้สามารถบันทึกอาการและผลข้างเคียงที่เกิดขึ้นหลังการใช้ยา ซึ่งช่วยให้สามารถติดตามผลของยาและประเมินความปลอดภัยในการใช้ยาได้อย่างเป็นระบบ
    \item การออกแบบ UI ที่ใช้งานง่าย เหมาะกับผู้สูงอายุ พร้อมระบบเสียงอ่านข้อมูลยา (Text-to-Speech)
    \item แอปพลิเคชันสามารถเก็บประวัติการใช้ยาและการแจ้งเตือนต่าง ๆ ในเครื่อง ทำให้ผู้ใช้สามารถเรียกดูย้อนหลัง
\end{enumerate}

